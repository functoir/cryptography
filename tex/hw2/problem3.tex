\begin{problem}
Let $a,b \in \Z$.
\begin{enumerate}\renewcommand{\itemsep}{3mm}
\item Let $\gcd(a,b)=g \neq 0$.  Prove that $\gcd(a/g,b/g)=1$.

\noindent
\begin{Answer}

By factorization let:

\begin{equation}
  a=\prod\, i^{a_{i}}, \;b =\prod\, i^{b_{i}} \;\forall i \in \ZZ^{+}
\end{equation}

\noindent

Then, since the $\gcd$ of two numbers must divide both numbers:

\begin{equation}
  g=\gcd(a,b)=\prod i^{\min\{a_{i}, b_{i}\}} \;\forall i \in \ZZ^{+}
\end{equation}

And;

\begin{equation}
  \frac{a}{g}=\prod i^{a_{i}-\min\{a_{i}, b_{i}\}}, \;\frac{b}{g} =\prod i^{b_{i}-\min\{a_{i}, b_{i}\}} \;\forall i \in \ZZ^{+}
\end{equation}

However, we know that:

\begin{equation}
  \min\{a_{i}, b_{i}\} = a_{i},\, \mathbf{OR} \;\min\{a_{i}, b_{i}\} = b_{i} \;\forall i \in \ZZ^{+}
\end{equation}

The equation (4) implies that \textit{at least} one of the following is true:

\begin{equation}
  a_{i}-\min\{a_{i}, b_{i}\} = 0, \mathbf{OR} \;b_{i}-\min\{a_{i}, b_{i}\} = 0 \;\forall i \in \ZZ^{+}
\end{equation}

Thus:

\begin{equation}
  \gcd(\frac{a}{g},\frac{b}{g})=\prod \, i^{\min\{0, a_{i}, b_{i}\}} \;\forall i \in \ZZ^{+}
\end{equation}

\begin{equation}
  \gcd(\frac{a}{g},\frac{b}{g})=\prod \, 1 \;\forall i \in \ZZ^{+}
\end{equation}

\begin{equation}
  \gcd(\frac{a}{g},\frac{b}{g})=1
\end{equation}
\end{Answer}
\newpage
\item Prove that $\gcd(a+kb,b)=\gcd(a,b)$ for all $k \in \Z$.
  We stated this in lecture, but now you can check it carefully
  yourself!

\noindent
\begin{Answer}
\setcounter{equation}{0}

Let $g_{1} = \gcd(a, b)$ and $g_{2} = \gcd(a+kb, b)$

By definition of $g_{1}$ as the $\gcd$ of $a$ and $b$:

\begin{equation}
g_{1}\, |\, a \;\text{and}\; g_{1}\, |\, b \Rightarrow \exists x, y \in \Z^{+} \, \backepsilon\, a = g_{1}x \;\text{and}\, b = g_{1}y
\end{equation}

Similarly, by definition of $g_{2}$ as the $\gcd$ of $a+kb$ and $b$:
\begin{equation}
g_{2}\, |\, (a+kb) \;\text{and}\; g_{2}\, |\, b \Rightarrow \exists m, n \in \Z^{+} \, \backepsilon\, (a+kb) = g_{2}m \;\text{and}\, b = g_{2}n
\end{equation}

We \textit{know} that $g_{2}$ divides $b$ (by definition of it being the $\gcd$ of $a+kb$ and $b$).

Therefore, subtracting a multiple of $g_{2}$ (in this case, $b$), from $a + kb$ does not change the divisibility status of result.

In particular,

\begin{equation}
  \zaff{g_{2}\, |\, b\, \land\, g_{2}\, |\, (a + kb)} \Rightarrow g_{2}\, |\, (a+kb-b) \Rightarrow g_{2}\, |\, (a+kb-2b) \Rightarrow \dots \Rightarrow \crim{g_{2}\,|\,(a + kb-kb) \Rightarrow g_{2}\,|\,a}
\end{equation}

Similarly, we know that $g_{1}$ divides both $a$ and $b$ (by definition of it being the $\gcd$ of the two numbers).
Therefore, adding a multiple of $g_1$ (in this case, $b$) to $a$ does not change the divisibility status.

In particular,

\begin{equation}
  \zaff{g_{1}\, |\, a\, \land\, g_{1}\, |\, b} \Rightarrow g_{1}\, |\, (a+b) \Rightarrow g_{1}\, |\, (a+2b) \Rightarrow \dots \Rightarrow \crim{g_{1}\,|\,(a + kb)}
\end{equation}

Thus, we have shown that $g_{1}\, |\, (a + kb)$ and $g_{2}\, |\, a$.

\crim{However, we earlier defined $g_{1} = \gcd(a, b)$ and $g_{2} = \gcd(a+kb, b)$}.

This implies:

\begin{equation}
  g_{1} \le g_{2} \land g_{2} \le g_{1} \Rightarrow g_{1} = g_{2}
\end{equation}
\end{Answer}

\end{enumerate}

\end{problem}

