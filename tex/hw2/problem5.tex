\begin{problem}
Let $f_0=f_1=1$ and $f_{i+1}=f_i+f_{i-1}$ for $i \geq 1$ denote the Fibonacci numbers.  
\begin{enumerate}\renewcommand{\itemsep}{3mm}
\item Use the Euclidean algorithm to show that $\gcd(f_i,f_{i-1})=1$
for all $i \geq 1$. (Again, we did this quickly in lecture, but now do
it carefully!)

\begin{Answer}
\setcounter{equation}{0}
For arbitrary $i\,\in\,\ZZ^+$,

\begin{equation}
  f_{i+1} = f_{i} + f_{i-1} \; \text{(by definition of the Fibonacci numbers)}
\end{equation}

Let $g$ be the greatest common divisor of $f_{i}+1$ and $f_{i}$.,

\begin{equation}
  g\, |\, f_{i}\, \land\, g\, |\, f_{i+1} \Rightarrow g\, |\, (f_{i+1} - f_{i}) = f_{i-1}
\end{equation}

\begin{equation}
  g\, |\, f_{i-1}\, \land\, g\, |\, f_{i} \Rightarrow g\, |\, (f_{i} - f_{i-1}) = f_{i-2}
\end{equation}

\begin{equation}
  g\, |\, f_{i-2}\, \land\, g\, |\, f_{i-1} \Rightarrow g\, |\, (f_{i-1} - f_{i-2}) = f_{i-3}
\end{equation}

\begin{equation*}
\vdots
\end{equation*}

\begin{equation}
  g\, |\, f_{3}\, \land\, g\, |\, f_{2} \Rightarrow g\, |\, (f_{3} - f_{2}) = f_{1}
\end{equation}

\begin{equation}
  g\, |\, f_{2}\, \land\, g\, |\, f_{1} \Rightarrow g\, |\, (f_{2} - f_{1}) = f_{0} = 1
\end{equation}

\begin{equation}
  g \, |\, 1 \Rightarrow g = 1
\end{equation}

\end{Answer}

\item Find $\gcd(11111111,11111)$.

\begin{Answer}
\setcounter{equation}{0}

Using the Euclidean algorithm:

\begin{equation}
  \gcd(11111111,11111) = \gcd(11111, 111)
\end{equation}

\begin{equation}
  \gcd(11111, 111) = \gcd(111, 11)
\end{equation}

\begin{equation}
  \gcd(111, 11) = \gcd(11, 1)
\end{equation}

\begin{equation}
  \gcd(11, 1) = \gcd(1, 0) = 1
\end{equation}
\end{Answer}

\newpage
\item Let $a=111\cdots 11$ be formed with $f_i$ repeated $1$s and let $b=111\cdots 11$ be formed with $f_{i-1}$ repeated $1$s.
Find $\gcd(a,b)$.  \\\emph{[Hint: Compare your computations in parts (a) and (b).]}

\begin{Answer}
\setcounter{equation}{0}

Let $S = \{s_{1}, s_{2}, s_{3}, \dots\} \subset \ZZ$ be the set of all integers formed by some $i$ repeated $1$s.

\noindent
For arbitrary $i$ and $j\, \in \ZZ^+$ such that $i \ge j$, $s_{i} \mod s_{j} = s_{i \mod j}$

\begin{proof}
\vspace{3pt}
\indent
In $S$, multiplying elements by positive integer powers of $10$ is equivalent to a shift of the 1s in the number.
For instance $11 \cdot 10 = 110$. This does not map elements in $S$ into $S$,
but it allows us to deduct deduct these new shifted values from elements in $S$ and
get rid of leading $1$s. For two powers $i$ and $j$ such that $i \ge j$,
we can repeatedly erase the first $j$ $1$s in $s_{i}$ by deducting a shifter version of $s_{j}$,
until we are no longer able to (at which point the remaining value has less $1$s than $s_{j}$.)
This is analogous to taking the $\mod$ of $i$ by $j$ and defining the remainder, $i \mod j$,
to be the number of $1$s that we get after fully erasing occurrences of $s_{j}$ in $s_{i}$.

\noindent
A good example can be seen in (b) above.
Here is another example:

\begin{equation}
  11111111 \equiv 11  \pmod{111}
\end{equation}

\noindent
Or, generally:

\begin{equation}
  s_{8} \mod s_{3} \equiv s_{(8\pmod{3})} = s_{2} \pmod{s_{3}}
\end{equation}
\end{proof}

Thus, following from part (a) above:

\begin{equation}
  \gcd(s_{f_{i+1}}, s_{f_{i}}) = \gcd(s_{f_{i}}, s_{f_{i-1}}) = \cdots = \gcd(s_{f_{1}}, s_{f_{0}}) = \gcd(1, 1) = 1
\end{equation}

\end{Answer}
\end{enumerate}
\end{problem}
