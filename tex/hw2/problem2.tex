
\begin{problem}
Let $n \geq 3$ and $S_n$ be the symmetric group on $\{1,\dotsc,n\}$.
We say that $\sigma \in S_n$ \emph{has a fixed point} if there exists
$k \in \{1,\dotsc,n\}$ such that $\sigma(k)=k$.  Prove that the
probability that a random $\sigma \in S_n$ has a fixed point is $\geq
5/8$ and $\leq 2/3$.  Here, ``probability'' means that the number of
those permutations with a fixed point divided by the number of all
permutations.  (Conclude that a random substitution cipher, realized
as a random permutation in $S_n$, is likely to fix at least one
symbol.)
\end{problem}
  

\begin{Answer}

Let $p(n)$ define the probability that a random $\sigma \in S_n$ has some fixed point.
Then:

\begin{equation}
p(n) = p(\text{single fixed point}) + p(\text{$2$ fixed points}) + \cdots + p(\text{$n$ fixed points})
\end{equation}

\noindent
We also have to account for permutations that have multiple fixed points.

\noindent
While isolating repeated points, some are deducted twice and have to be re-added into the sum.

\begin{equation}
p(n) = {n \choose 1} \cdot (n - 1)! - {n \choose 2} \cdot (n - 2)! + {n \choose 3} (n - 3)! -\cdots \pm {n \choose n} \cdot (n - n)!
\end{equation}

\noindent
We can expand this sequence into an alternating series.

\noindent
We also have to remember to divide by the numner of all possible permutations because we want the probability and not the counts.

\begin{equation}
p(n) = \frac{\overset{n}{\underset{k=1}{\sum}} {(-1)}^{k + 1} \frac{n!}{k!(n-k)!}\cdot(n-k)!}{n!} = \overset{n}{\underset{k=1}{\sum}} \frac{(-1)^{k + 1}}{k!}
\end{equation}

\begin{equation}
p(3) = 1 - \frac{1}{2!} + \frac{1}{3!} = \frac{2}{3} \approx 0.6666
\end{equation}

\begin{equation}
p(4) = 1 - \frac{1}{2!} + \frac{1}{3!} - \frac{1}{4} = \frac{5}{8} = 0.625
\end{equation}

\begin{equation}
p(5) = 1 - \frac{1}{2!} + \frac{1}{3!} - \frac{1}{4} + \frac{1}{5!} = \frac{19}{30} \approx 0.6333
\end{equation}

\begin{equation}
p(6) = 1 - \frac{1}{2!} + \frac{1}{3!} - \frac{1}{4} + \frac{1}{5!} - \frac{1}{6!} = \frac{91}{144} \approx 0.6319
\end{equation}

\begin{equation}
p(7) = 1 - \frac{1}{2!} + \frac{1}{3!} - \frac{1}{4} + \frac{1}{5!} - \frac{1}{6!} + \frac{1}{7!} = \frac{177}{280} \approx 0.6321
\end{equation}

\begin{equation}
p(8) = 1 - \frac{1}{2!} + \frac{1}{3!} - \frac{1}{4} + \frac{1}{5!} - \frac{1}{6!} + \frac{1}{7!} - \frac{1}{8!} = \frac{3641}{5760} \approx 0.6321
\end{equation}

\begin{equation}
p(9) = 1 - \frac{1}{2!} + \frac{1}{3!} - \frac{1}{4} + \frac{1}{5!} - \frac{1}{6!} + \frac{1}{7!} - \frac{1}{8!} + \frac{1}{9!} \approx 0.6321
\end{equation}

\begin{equation*}
\vdots
\end{equation*}

\noindent
\crim{After expanding a few terms of the sequence,
we see that the series queickly converges to $p(n) \approx 0.6321$.}
\end{Answer}
