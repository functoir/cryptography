\documentclass[11pt]{amsart}
\usepackage{amssymb,amsthm,amsmath,amstext}
\usepackage{mathdots}

\usepackage[left=1.2in,right=1.2in,top=1.5in,bottom=1.5in,nohead,nofoot]{geometry}

\newcommand{\Z}{\mathbb{Z}}

\pagestyle{empty}

\newcounter{problem}
\setcounter{problem}{0}

\newenvironment{problem}[1][]%
{%
\stepcounter{problem} \vspace{.2cm} \noindent {\bf \arabic{problem}.} {\it #1}~ %
}{%
\vspace{.2cm}%
} 

\renewcommand{\theenumi}{\alph{enumi}}

\begin{document}

\thispagestyle{empty}

\title{} 

%\date{\today}

\noindent{\sc
Dartmouth College Department of Mathematics \\ \bf
Math 75 Cryptography \\ \rm
Spring 2022\\[2ex]
Problem Set \# 2 (upload to Canvas by Friday, April 15, 10:10 am EDT)}

\bigskip

\noindent {\bf Problems:}

\begin{problem}
A disadvantage of the general substitution cipher is that both sender and receiver must commit the permuted cipher sequence to memory.  A common technique for avoiding this is to use a keyword from which the cipher sequence can be generated.  For example, using the keyword \texttt{CIPHER}, write out the keyword followed by unused letters in normal order and match this against the plaintext letters:

\begin{verbatim}
  plain:   a b c d e f g h i j k l m n o p q r s t u v w x y z
  cipher:  C I P H E R A B D F G J K L M N O Q S T U V W X Y Z
\end{verbatim}

If it is felt that this process does not produce sufficient mixing, write the remaining letters on successive lines and then generate the sequence by reading down the columns:

\begin{verbatim}
                        C I P H E R
                        A B D F G J
                        K L M N O Q
                        S T U B W X
                        Y Z
\end{verbatim}

This yields the sequence: \verb|C A K S Y I B L T Z P D M U H F N V E G O W R J Q X|.

Such a system is used in the following decoded ciphertext: 

\begin{verbatim}
       UZQSOVUOHXMOPVGPOZPEVSGZWSZOPFPESXUDBMETSXAIZ
       itwasdisclosedyesterdaythatseveralinformalbut

       VUEPHZHMDZSHZOWSFPAPPDTSVPQUZWYMXUZUHSX
       directcontactshavebeenmadewithpolitical

       EPYEPOPDZSZUFPOMBZWPFUPZHMDJUDTMOHMQ
       representativesofthevietconginmoscow
\end{verbatim}

Determine the keyword.

\end{problem}


\begin{problem}
Let $n \geq 3$ and $S_n$ be the symmetric group on $\{1,\dotsc,n\}$.
We say that $\sigma \in S_n$ \emph{has a fixed point} if there exists
$k \in \{1,\dotsc,n\}$ such that $\sigma(k)=k$.  Prove that the
probability that a random $\sigma \in S_n$ has a fixed point is $\geq
5/8$ and $\leq 2/3$.  Here, ``probability'' means that the number of
those permutations with a fixed point divided by the number of all
permutations.  (Conclude that a random substitution cipher, realized
as a random permutation in $S_n$, is likely to fix at least one
symbol.)
\end{problem}

\newpage

\begin{problem}
Let $a,b \in \Z$.
\begin{enumerate}\renewcommand{\itemsep}{3mm}
\item Let $\gcd(a,b)=g \neq 0$.  Prove that $\gcd(a/g,b/g)=1$.
\item Prove that $\gcd(a+kb,b)=\gcd(a,b)$ for all $k \in \Z$.
\end{enumerate}
We stated this in lecture, but now you can check it carefully
yourself!
\end{problem}


\begin{problem}
Compute some inverses!
\begin{enumerate}\renewcommand{\itemsep}{3mm}
\item Use the extended Euclidean algorithm to compute $367^{-1}$ in $(\Z/1001\Z)^\times$ and $1001^{-1}$ in $(\Z/367\Z)^\times$.  [Do this by hand.]
\item Compute $314159265^{-1}$ in $(\Z/2718281828\Z)^\times$.  [You may use a computer!]
\end{enumerate}
\end{problem}


\begin{problem}
Let $f_0=f_1=1$ and $f_{i+1}=f_i+f_{i-1}$ for $i \geq 1$ denote the Fibonacci numbers.  
\begin{enumerate}\renewcommand{\itemsep}{3mm}
\item Use the Euclidean algorithm to show that $\gcd(f_i,f_{i-1})=1$
for all $i \geq 1$. (Again, we did this quickly in lecture, but now do
it carefully!)
\item Find $\gcd(11111111,11111)$.
\item Let $a=111\cdots 11$ be formed with $f_i$ repeated $1$s and let $b=111\cdots 11$ be formed with $f_{i-1}$ repeated $1$s.  Find $\gcd(a,b)$.  \\\emph{[Hint: Compare your computations in parts (a) and (b).]}
\end{enumerate}
\end{problem}



\vfill

\end{document}
