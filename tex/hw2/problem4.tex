\begin{problem}
Compute some inverses!
\begin{enumerate}\renewcommand{\itemsep}{3mm}
\item Use the extended Euclidean algorithm to compute $367^{-1}$ in $(\Z/1001\Z)^\times$ and $1001^{-1}$ in $(\Z/367\Z)^\times$.  [Do this by hand.]
\begin{Answer}
\setcounter{equation}{0}

\item $367^{-1}$ in $(\ZZ/1001\ZZ)^\times$

First, we can use the Euclidean algorithm to compute the $\gcd$ (and decompose the coefficients).
\begin{equation}
  x\,\equiv\,367^{-1} \pmod {1001}
\end{equation}
\begin{equation}
  1001 = 2 \cdot 367 + 267
\end{equation}
\begin{equation}
  367 = 1 \cdot 267 + 100
\end{equation}
\begin{equation}
  267 = 2 \cdot 100 + 67
\end{equation}
\begin{equation}
  100 = 1 \cdot 67 + 33
\end{equation}
\begin{equation}
  67 = 2 \cdot 33 + 1
\end{equation}

\noindent
We can now use the extended Euclidean algorithm to back-substitute and compute the inverse.
\begin{equation}
  1 = 67 - 2 \cdot 33
\end{equation}
\begin{equation}
  1 = 100 - 3 \cdot 33
\end{equation}
\begin{equation}
  1 = -2 \cdot 100 + 3 \cdot 67
\end{equation}
\begin{equation}
  1 = -8 \cdot 100 + 3 \cdot 267
\end{equation}
\begin{equation}
  1 = -8 \cdot 367 + 11 \cdot 267
\end{equation}
\begin{equation}
  1 = -30 \cdot 367 + 11 \cdot 1001
\end{equation}
\begin{equation}
  1 \equiv -30 \cdot 367 + 11 \cdot 1001 \pmod {1001}
\end{equation}
\begin{equation}
  1 \equiv -30 \cdot 367 \pmod {1001}
\end{equation}
\begin{equation}
  1 \equiv 971 \cdot 367 + 11 \cdot 1001 \pmod {1001}
\end{equation}

\noindent
\crim{Thus: $367^{-1}$ in $(\ZZ/1001\ZZ)^\times = 971$}

\newpage
\setcounter{equation}{0}
\item $1001^{-1}$ in $(\Z/367\Z)^\times$

First, we can use the Euclidean algorithm to compute the $\gcd$ (and decompose the coefficients).
\begin{equation}
  x\,\equiv\,1001^{-1} \pmod{367} \equiv 267 \pmod{367}
\end{equation}
\begin{equation}
  367 = 1 \cdot 267 + 100
\end{equation}
\begin{equation}
  267 = 2 \cdot 100 + 67
\end{equation}
\begin{equation}
  100 = 1 \cdot 67 + 33
\end{equation}
\begin{equation}
  67 = 2 \cdot 33 + 1
\end{equation}

\noindent
We can now use the extended Euclidean algorithm to back-substitute and compute the inverse.
\begin{equation}
  1 = 67 - 2 \cdot 33
\end{equation}
\begin{equation}
  1 = 100 - 3 \cdot 33
\end{equation}
\begin{equation}
  1 = -2 \cdot 100 + 3 \cdot 67
\end{equation}
\begin{equation}
  1 = -8 \cdot 100 + 3 \cdot 267
\end{equation}
\begin{equation}
  1 = -8 \cdot 367 + 11 \cdot 267
\end{equation}
\begin{equation}
  1 \equiv -8 \cdot 367 + 11 \cdot 267 \pmod{367} \equiv 11 \pmod{367}
\end{equation}

\noindent
\crim{Thus: $1001^{-1}$ in $(\ZZ/367\ZZ)^\times = 11$}

\end{Answer}
\newpage
\item Compute $314159265^{-1}$ in $(\Z/2718281828\Z)^\times$.  [You may use a computer!]
\end{enumerate}
\end{problem}

\begin{Answer}

My first approach, which was really slow, was to count up from $1$ to the base,
checking if any number multiplied by the target to give a product congruent to $1$.

\color{crimson}
\begin{Verbatim}[commandchars=\\\{\}]

\teal{-- | Compute the inverse of a number modulo N}
invN :: Integral p => p -> p -> p  \teal{-- Type must be integral (i.e. support div, mod)}
invN _ 0 = 0
invN _ 1 = 1
invN base a = iter base a 1
  where

    \teal{-- | Iterate from 1 to n, checking every number.}
    iter n a b
    | a > n || b > n = 0
    | a `mul` b == 1 = b
    | otherwise = iter n a (b + 1)
  
    mul = mulN n
    \teal{-- | Compute multiplication modulo N.}
    \teal{--  The result is always in the range [0,N).}
    mulN :: Integral a => a -> a -> a -> a
    mulN base a b = (a * b) `mod` base
\end{Verbatim}
\color{black}

This program was not very promising ( it was doing too much unnecessary computation).

My next approach was to write a version of the euclidean algorithm that
tracks Bezout's coefficients as it runs the Euclidean algorithm.
\color{crimson}
\begin{Verbatim}[commandchars=\\\{\}]

\teal{-- | Run the Euclidean algorithm while tracking Bezout's coefficients}
euclid :: Integral b => b -> b -> (b, b, b)
euclid 0 y = (y, 0, 1)
euclid x y =
  let (gcd, x', y') = euclid (y `mod` x) x
      x = y' - (y `div` x) * x'
      y = x'
  in (gcd, x, y)

\teal{--- Results}
\teal{number: 2718281828}
\teal{number: 314159265}
\teal{(1,-143182048,1238891233) --> (gcd, coefficient 1, coefficient 2)}

\end{Verbatim}
\color{black}

Therefore, the inverse of $314159265$ in $(\ZZ/2718281828\ZZ)^\times$ is $1238891233$.

\end{Answer}
