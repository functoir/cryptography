
\begin{problem}
\noindent
The following message was encrypted using a simple substitution cipher:
\begin{verbatim}
53ddc305))6*;4826)4d.)4d);806*;48c8p60))85;;]8*;:d*8c83
(88)5*c;46(;88*96*?;8)*d(;485);5*c2:*d(;4956*2(5*-4)8p8*
;4069285);)6c8)4dd;1(d9;48081;8:8d1;48c85;4)485c528806*81
(d9;48;(88;4(d?34;48)4d;161;:188;d?;
\end{verbatim}
Decrypt the message.  \emph{Hint. Use frequency analysis: consider e,
ee, the, \ldots}
\end{problem}

\begin{Answer}
For fun, I tried to solve this problem using a brute-force algorithm --- turns out,
even when I limit branching to a factor of $2$, for a search depth of $20$ letters the search tree
grows to an order of $2^{20} = 1048576$.
\noindent
I soon resolved to frequency analysis and more informed substitutions.
To be able to use the tool shared in class, I substituted the symbols
in the ciphertext for English letters, starting from `A' and matching
all the symbols and letters as they are encountered:
\newline
\noindent

\color{blue}
\begin{verbatim}
ABCCDBEAFFGHIJKSGFJCLFJCFIKEGHIJKDKMGEFFKAINCKHOCPIOCHKDKBPKKF
AHDIJGPIKKHRGHQIKFHCPIJKAFIAHDSOHCPIJRAGHSPAHTJFKMKHIJEGRSKAFI
FGDKFJCCINPCRIJKEKNIKOKCNIJKDKAIJFJKADASKKEGHKNPCRIJKIPKKIJPCQ
BJIJKFJCINGNIONKKICQI
\end{verbatim}
\color{black}
Looking at the frequency analysis
\color{black}
- The most common 3-letter sequence is ``IJK''. In English, this is
usually ``the''. Substitute in the three letters.
\color{blue}
\begin{Verbatim}[commandchars=\\\{\}]
ABCCDBEAFFGH\textcolor{crimson}{the}SGFhCLFhCFteEGH\textcolor{crimson}{the}DeMGEFFeAtNCPtOCHeDeBPeeF
AHDthGPteeHRGHQteFHCP\textcolor{crimson}{the}AFtAHDSOHCPthRAGHSPAHThFeMeHthEGRSeAFt
FGDeFhCCtNPCR\textcolor{crimson}{the}EeNteOeCN\textcolor{crimson}{the}DeAthFheADASeeEGHeNPCR\textcolor{crimson}{the}tPeethPCQ
Bh\textcolor{crimson}{the}FhCtNGNtONeetCQt
\end{Verbatim}
\color{black}

\noindent
The most common letter in the ciphertext is ``K''. In English, this usually corresponds
to the letter `e'. This further validates the above exchange.
\newline
- The $4$th, $5$th, and $6$th most common letters in the ciphertext are
`C', `F', and `H'. In English, the next $3$ unused letters are
`a', `i', and `o'.
   Comparing these options: (a) `C' appears in a pair, so it is more likely to be `o'
   since English does not have many occurrences of ``ii'' or ``aa''. (b) `F' appears
   before `t' on numerous occassions, so it might be either `s' or `n'.
   Let's pick `s', since it is more common.

\color{blue}
\begin{Verbatim}[commandchars=\\\{\}]
AB\textcolor{crimson}{oo}DBEA\crim{ss}GHtheSG\crim{s}h\crim{o}L\crim{s}h\crim{os}teEGHtheDeMGE\crim{ss}eAtN\crim{o}PtO\crim{o}HeDeBPees
AHDthGPteeHRGHQtesH\crim{o}PtheA\crim{s}tAHDSOH\crim{o}PthRAGHSPAHTh\crim{s}eMeHthEGRSeA\crim{s}t
\crim{s}GDe\crim{s}h\crim{oo}tNP\crim{o}RtheEeNteOe\crim{o}NtheDeAth\crim{s}heADASeeEGHeNP\crim{o}RthetPeethP\crim{o}Q
Bhthe\crim{s}h\crim{o}tNGNtONeet\crim{o}Qt
\end{Verbatim}
\color{black}
`E' completes the word ``hosteE'' --- it is most likely `l'.
Another common two-letter pair is ``GH''. In English, this could be ``to'' or ``in''.
However, it seems to appear frequently inside other words so it most likely is ``in''.
\color{blue}
\begin{Verbatim}[commandchars=\\\{\}]
ABooDBlAss\crim{in}theS\crim{i}shoLshoste\crim{lin}theDeM\crim{il}sseAtNoPtOo\crim{n}eDeBPees
AnDth\crim{i}Ptee\crim{n}R\crim{in}Qtes\crim{n}oPtheAstA\crim{n}DSO\crim{n}oPthRA\crim{in}SPA\crim{n}ThseMe\crim{n}th\crim{li}RSeAst
s\crim{i}DeshootNPoRthe\crim{l}eNteOeoNtheDeAthsheADASee\crim{lin}eNPoRthetPeethPoQ
BhtheshotN\crim{i}NtONeetoQt
\end{Verbatim}
\color{black}

\noindent
Next, `D' and `M' appear to complete the word ``DeMil'', which can be ``devil''.
\newline
Next, `B' and `A' complete ``BlAss'', and `B' also completes `Bood'.
This combo is most likely `g' and `a'.
\color{blue}
\begin{Verbatim}[commandchars=\\\{\}]
\crim{ag}ood\crim{g}l\crim{a}ssintheSishoLshostelinthe\crim{d}e\crim{v}ilsse\crim{a}tNoPtOone\crim{d}e\crim{g}Pees
\crim{a}n\crim{d}thiPteenRinQtesnoPthe\crim{a}st\crim{a}n\crim{d}SOnoPthR\crim{a}inSP\crim{a}nThse\crim{v}enthliRSe\crim{a}st
si\crim{d}eshootNPoRtheleNteOeoNthede\crim{a}thshe\crim{a}d\crim{a}SeelineNPoRthetPeethPoQ
\crim{g}htheshotNiNtONeetoQt
\end{Verbatim}
\color{black}
\noindent
`P' completes ``degPees'' --- it is most likely `r'.
\newline
`S' and `L' complete ``SishoLs' --- they're most-likely `b' and `p', respectively.
\newline
`Q' completes ``throQgh'' --- it is most likely `u'.
\newline
`R' completes ``Rinutes'' --- it is most likely `m'.
\newline
`O' and `T' complete ``bOnorthmainbranTh'' --- they're clearly
`y' and `c'.
\newline
`N' completes ``Nrom the'' --- it is most likely `f'.
\newline
\newpage
The decoded message becomes:
\color{crimson}
\begin{verbatim}
agoodglassinthebishopshostelinthedevilsseatfortyonedegrees
andthirteenminutesnortheastandbynorthmainbranchseventhlimbeast
sideshootfromthelefteyeofthedeathsheadabeelinefromthetree
throughtheshotfiftyfeetout
\end{verbatim}
\color{black}
When spaced out:
\color{crimson}
\begin{verbatim}
  A good glass in the bishop's hostel in the devil's seat
  forty-onedegrees and thirteen minutes northeast and by
  north main branchseventh limb east side, shoot from the
  left eye of the death's heada bee-line from the tree
  through the shot fifty feet out.
  \end{verbatim}
\end{Answer}
