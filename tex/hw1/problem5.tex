\begin{problem}
For fun, take a stab at this problem.  In one of his cases, Sherlock Holmes was confronted with the following message:

\begin{center}
534 C2 13 127 36 31 4 7 21 41

DOUGLAS 109 293 5 37 BIRLSTONE

26 BIRLSTONE 9 127 171
\end{center}
Although Watson was puzzled, Holmes was immediately able to deduce the type of cipher.  Can you?
\begin{Answer}
A possible explanation is a substitution scheme where entire words
are substituted for specific numbers --- for instance,
``534'' might be cryptic for ``The queen'', etc.
This is perhaps less susceptible to the frequency analysis on letters,
but individual words also have disparities on relative frequencies in most
languages, so frequency analysis might still be able to decipher meaning
behind the numbers.

\noindent
It might also be referencing a specific location or book, with ``DOUGLAS'' and ``BIRLSTONE''
being some kind of identifier, such as the outhor, and the numbers telling where to look in the book.

\noindent
This is a tougher system to scale --- one would have to send the same book to everyone he wants to 
communicate with, in which case others might know about it, buy the same book, and learn how to
 decode the cryptic series of words and numbers.

\end{Answer}
\end{problem}
