
\begin{problem}
In one of Dorothy Sayers' mysteries, Lord Peter is confronted with the following message:

\vspace{1ex}

\begin{verbatim}
I thought to see the fairies in the fields, but I saw only the evil
elephants with their black backs.  Woe! how that sight awed me!  The
elves danced all around and about while I heard voices calling clearly.
Ah! how I tried to see--throw off the ugly cloud--but no blind eye of a
mortal was permitted to spy them.  So then came minstrels, having gold
trumpets, harps and drums.  These played very loudly beside me,
breaking that spell.  So the dream vanished, whereat I thanked Heaven.  I
shed many tears before the thin moon rose up, frail and faint as a sickle of
straw.  Now though the Enchanter gnash his teeth vainly, yet shall he
return as the spring returns.  Oh, wretched man!  Hell gapes, Erebus now
lies open.  The mouths of Death wait on thy end.
\end{verbatim}
\vspace{1ex}

He also discovers the key to the message, which is a sequence of integers:
\[ 
\texttt{7876565434321123434565678788787656543432112343456567878878765654433211234}
\]

\begin{enumerate}\renewcommand{\itemsep}{3mm}
\item Decrypt the message. \emph{Hint. What is the largest integer value?}

\begin{Answer}
The largest integer value is $8$.
All integer values in the key are non-zero, so it's unlikely that the mapping field includes 0.
Using this strategy, I wrote a simple program that split the message into 8 lists of letters
resulting in a matrix with $73$ columns and $8$ rows: 

{
\color{blue}%
\begin{verbatim}
ITFNDWEHHAWAWECOBEOLRISOLUDOPDECTIRAREEYEGLRSEKNNETRAICRONGTNAUSTRAPSOOEO
TOATSOVATCOTELEUOIIILTEFYTERETMARNURUPRBBTLEHAEIYFHOINKAUCNELLRPUENENPUAN
HSIHBNINHKESDVDNUHCNYREFCNYTROSMEGMPMLYERHSAETDSTOISLTLWGHAEYLNRRTHSOETTT
OEREULLTEBHIMEADTEEGAITTLOEAMSOELGPSSALSEAOMDIHHERNEAAENHASTYHAINCEEWNHHH
UEIFTYESIAOGESLAWASCHEHHOBOLIPTMSOEATYOIATTVWTEEAEMUNSOOTNHHEESNSHLRLTSWY
GTEIITLWRCWHTDLNHRCLHDREULFWTYHIHLTNHEUDKSHAHHADRTOPDAFWHTHVTRTGOELEIHOAE
HHSESHEIBKTTHAADIDAEOTOUDIAATTENADSDEDDEIPENEAVMSHOFFSSTEEIASEHRHDGBEEFIN
TEILAEPTLSHAENRALVLAWOWGBNMSEHNSVTHDSVLMNEDIRNEABENRAITHERSIHTEEWMAUSMDTD
\end{verbatim}
}

\noindent
Using the key to pick out rows for each column --- or word position in the sentence ---
I picked out the following message:

{\color{crimson}
\begin{verbatim}
  hesittethbetweenthecherubimstheisles
  maybegladthereofastheriversinthesouth
\end{verbatim}
}
\noindent
When spaced out properly, the above reads:
{\color{crimson}
\begin{verbatim}
  He sitteth between the cherubims the isles
  may be glad thereof as the rivers in the south
\end{verbatim}
}

\end{Answer}
\item If the algorithm is known, but not the key, how secure is this
encryption scheme?

\begin{Answer}
This scheme employs arbitrary substitutions to encrypt the message.
This is more secure than other methods such as standard substitution ciphers.

\noindent
Even if the algorithm is known, but not the key, then there are at least $8^{n}$ possibilities
(assuming the interceptor knows enough to map the letters over $8$-space), or $26^{n}$
if the interceptor does not know enough information to map the letters over the reduced space.

\noindent
These are both negligible possibilities --- it would take immense computational power to
decrypt such a message.
However, that's perhaps \textit{still} not as secure as modern methods 
including \textbf{RSA} and \textbf{AES}.
\end{Answer}

\item If the key is known, but not the algorithm, how secure is this
encryption scheme?

\begin{Answer}
  A problem with this method is that the key, if known, gives away important information about how
  the message was encrypted. For instance, we were able to reduce the complexity from $26^{n}$ to $8^{n}$
  just by exploiting a single fact about the encryption key.

\noindent
Knowing the full key made decrypting the entire message a trivial problem.
\end{Answer}
\end{enumerate}
\end{problem}
