\begin{problem}
  Substitution cipher.

  \color{zaffre}
  \begin{Verbatim}
    OVDKLJOODKBPLABZGUFVDKLACVQLKBZVWLJJVGVLYBZVZUJJQLKBZVJLYVSUYVDTL
    FVZVKDYOBZVJLRDJLYVCIZLJLFVZVKGUFVDKLACVGUFVDKLACVIUBZDIUJJQLKBZV
    CLYCLQLJOODKBPLABZQLKBZVODAGZBVKCLQODKBPLABZBZLAGZKLAYOBZVGUKOJVO
    VDKBZBZVRKLDPZVKCSVJJLYBZVPKVPDUYCBZVRZDFVBZVCBUJJYLKBZUYBZVUKZVD
    KBCBZVZUJJIUYOCUYBZVUKFVUYCDYOBZVGKDYUBVLQYVIZDPSCZUKVUYBZVUKPACW
    JVCDYOBZVUKTKDUYCDZUYBQLKWUSZVKCUHUCBZVLKOVKLQQUFVPLOAJLSUCCPDJJ
  \end{Verbatim}
  \color{black}
\end{problem}

\begin{Answer}

  O --- d

  V --- e

  D --- a

  K --- r

  L --- o

  J --- l


  \noindent
  I used this website to check word and letter frequencies:

  \noindent
  \color{crimson}
  \url{https://www3.nd.edu/~busiforc/handouts/cryptography/Letter%20Frequencies.html}
  \color{black}

  \noindent
  For a start, we can analyze the frequencies of different letters in the ciphertext.

  \begin{multicols*}{2}
    Standard English frequencies:
    \centering
    \begin{tabular}{||c | c ||}
      \hline
      letter & percentage frequency\\
      \hline
      \crim{e} & $12.702$\\
      \hline
      \crim{t} & $9.056$\\
      \hline
      a & $8.087$\\
      \hline
      o & $7.507$\\
      \hline
      i & $6.966$\\
      \hline
      n & $6.749$\\
      \hline
      s & $6.234$\\
      \hline
      \crim{h} & $6.094$\\
      \hline
      r & $5.987$\\
      \hline
      d & $4.253$\\
      \hline
      l & $4.094$\\
      \hline
      c & $2.781$\\
      \hline
      u & $2.758$\\
      \hline
      m & $2.587$\\
      \hline
      w & $2.36$\\
      \hline
      f & $2.228$\\
      \hline
      g & $2.015$\\
      \hline
      y & $1.974$\\
      \hline
      p & $1.929$\\
      \hline
      b & $1.493$\\
      \hline
      v & $0.978$\\
      \hline
      k & $0.772$\\
      \hline
      j & $0.153$\\
      \hline
      x & $0.15$\\
      \hline
      q & $9.6e-2$\\
      \hline
      z & $7.4e-2$\\
      \hline
        \hline
    \end{tabular}
    \flushleft{}

    Ciphertext frequencies:
    \centering
    \begin{tabular}{||c | c | c ||} 
    \hline
      letter & percentage frequency & plaintext\\
      \hline
      \crim{v} & $12.853470437017995$ & e\\
      \hline
      \crim{z} & $10.025706940874036$ & h\\
      \hline
      l & $8.740359897172237$ & a\\
      \hline
      \crim{b} & $8.483290488431876$ & t\\
      \hline
      k & $8.483290488431876$ & i\\
      \hline
      u & $7.455012853470437$ & \\
      \hline
      d & $6.169665809768637$ & \\
      \hline
      j & $5.912596401028278$ & \\
      \hline
      c & $5.3984575835475574$ & \\
      \hline
      y & $5.3984575835475574$ & \\
      \hline
      o & $4.113110539845758$ & \\
      \hline
      a & $2.827763496143959$ & \\
      \hline
      p & $2.570694087403599$ & \\
      \hline
      q & $2.570694087403599$ & \\
      \hline
      f & $2.056555269922879$ & \\
      \hline
      g & $2.056555269922879$ & \\
      \hline
      i & $1.2853470437017995$ & \\
      \hline
      s & $1.2853470437017995$ & \\
      \hline
      r & $0.7712082262210797$ & \\
      \hline
      w & $0.7712082262210797$ & \\
      \hline
      t & $0.5141388174807198$ & \\
      \hline
      h & $0.2570694087403599$ & \\
      \hline
    \end{tabular}
    \flushleft{}
  \end{multicols*}

  \noindent
  We expect similarities in the frequencies.
  
  \noindent
  For a sure start, we know the most-common $3$-letter word in English is `the'.
  We can use this to find the most-common 3-gram in the ciphertext,
  substitute it with `the', and check how it matches up to our table above.

  \begin{Verbatim}[commandchars=\\\{\}]
    \zaff{ghci> frequencies $ ngrams 3 ciphertext}
      BZV: 4.651162790697675
      KBZ: 1.550387596899225
      VDK: 1.550387596899225
  \end{Verbatim}

  \noindent
  The highest frequency is `BZV', matching to the English word `the'.
  We would expect `B' to match to `t', `Z' to match to `h', and `V' to match to `e'.
  This appears a bit off from the single-letter frequencies, but all letters matched are
  still in the top letters in the frequency tables.

  \noindent
  After substitution:

  \begin{Verbatim}[commandchars=\\\{\}]
    OeDKLJOODKtPLAthGUFeDKLACeQLKtheWLJJeGeLYthehUJJQLKtheJLYeSUYeDTL
    FeheKDYOtheJLRDJLYeCIhLJLFeheKGUFeDKLACeGUFeDKLACeIUthDIUJJQLKthe
    CLYCLQLJOODKtPLAthQLKtheODAGhteKCLQODKtPLAththLAGhKLAYOtheGUKOJeO
    eDKththeRKLDPheKCSeJJLYthePKePDUYCtheRhDFetheCtUJJYLKthUYtheUKheD
    KtCthehUJJIUYOCUYtheUKFeUYCDYOtheGKDYUteLQYeIhDPSChUKeUYtheUKPACW
    JeCDYOtheUKTKDUYCDhUYtQLKWUSheKCUHUCtheLKOeKLQQUFePLOAJLSUCCPDJJ
  \end{Verbatim}

  \noindent
  We can expect L to map to either `a' or `o', depending on frequencies.

  \noindent
  Next, if we print the 4-grams we see
  \begin{Verbatim}
    "Ythe": 1.2953367875647668
    "oKth": 1.2953367875647668
    "Kthe": 1.0362694300518134
    "Othe": 1.0362694300518134
    "QoKt": 1.0362694300518134
    "YOth": 1.0362694300518134
    "eDKo": 1.0362694300518134
    "heUK": 1.0362694300518134
    "theU": 1.0362694300518134
    "Cthe": 0.7772020725388601
  \end{Verbatim}

  \noindent
  ``aKth'' or ``oKth'' occurs frequently. We would expect this to be
  `r', as in the sequence "arth" or `orth", which is common in English. 

  \noindent
  Thus, we can replace `K' $\rightarrow$ `r'

  \begin{Verbatim}
    OeDroJOODrtPoAthGUFeDroACeQortheWoJJeGeoYthehUJJQortheJoYeSUYeDTo
    FeherDYOtheJoRDJoYeCIhoJoFeherGUFeDroACeGUFeDroACeIUthDIUJJQorthe
    CoYCoQoJOODrtPoAthQortheODAGhterCoQODrtPoAththoAGhroAYOtheGUrOJeO
    eDrththeRroDPherCSeJJoYthePrePDUYCtheRhDFetheCtUJJYorthUYtheUrheD
    rtCthehUJJIUYOCUYtheUrFeUYCDYOtheGrDYUteoQYeIhDPSChUreUYtheUrPACW
    JeCDYOtheUrTrDUYCDhUYtQorWUSherCUHUCtheorOeroQQUFePoOAJoSUCCPDJJ
  \end{Verbatim}

  \noindent
  Looking at the tri-grams again:
  
  \begin{Verbatim}
    "the": 4.651162790697675
    "eDr": 1.550387596899225
    "rth": 1.550387596899225
    "Drt": 1.2919896640826873
    "Qor": 1.2919896640826873
    "Yth": 1.2919896640826873
    "ort": 1.2919896640826873
    "Dro": 1.0335917312661498
    "Oth": 1.0335917312661498
    "UFe": 1.0335917312661498
  \end{Verbatim}

  \noindent
  We see `D' occurs in ``eDr'', ``Drt'', and ``Dro''.
  One letter unused that makes sense in all these contexts is `a'.
  Thus, we can swap `D' $\rightarrow$ `a'.

  \begin{Verbatim}
    OearoJOOartPoAthGUFearoACeQortheWoJJeGeoYthehUJJQortheJoYeSUYeaTo
    FeheraYOtheJoRaJoYeCIhoJoFeherGUFearoACeGUFearoACeIUthaIUJJQorthe
    CoYCoQoJOOartPoAthQortheOaAGhterCoQOartPoAththoAGhroAYOtheGUrOJeO
    earththeRroaPherCSeJJoYthePrePaUYCtheRhaFetheCtUJJYorthUYtheUrhea
    rtCthehUJJIUYOCUYtheUrFeUYCaYOtheGraYUteoQYeIhaPSChUreUYtheUrPACW
    JeCaYOtheUrTraUYCahUYtQorWUSherCUHUCtheorOeroQQUFePoOAJoSUCCPaJJ
  \end{Verbatim}

  \noindent
  We now see the sequence ``Oear''.
  This should be either ``bear'', ``dear'', ``hear'', ``fear'', ``near'', ``pear'', or ``wear''.
  Of those, ``dear'' tends to occur more often at the beginning of letters or addressed messeges.
  Thus, we can swap `O' $\rightarrow$ `d'. 

  \begin{Verbatim}
    dearoJddartPoAthGUFearoACeQortheWoJJeGeoYthehUJJQortheJoYeSUYeaTo
    FeheraYdtheJoRaJoYeCIhoJoFeherGUFearoACeGUFearoACeIUthaIUJJQorthe
    CoYCoQoJddartPoAthQorthedaAGhterCoQdartPoAththoAGhroAYdtheGUrdJed
    earththeRroaPherCSeJJoYthePrePaUYCtheRhaFetheCtUJJYorthUYtheUrhea
    rtCthehUJJIUYdCUYtheUrFeUYCaYdtheGraYUteoQYeIhaPSChUreUYtheUrPACW
    JeCaYdtheUrTraUYCahUYtQorWUSherCUHUCtheorderoQQUFePodAJoSUCCPaJJ
  \end{Verbatim}

  \noindent
  We have ``dearoJd''. This should probably be ``dear old'',
  So we can replace `J' $\rightarrow$ `l'.

  \noindent
  Looking at the ciphertext,
  We have the sequence ``the order oQ''.
  This is most-likely ``the order of'',
  so we can replace `Q' $\rightarrow$ `f'.


  \begin{Verbatim}
    dearolddartPoAthGUFearoACefortheWolleGeoYthehUllfortheloYeSUYeaTo
    FeheraYdtheloRaloYeCIholoFeherGUFearoACeGUFearoACeIUthaIUllforthe
    CoYCofolddartPoAthforthedaAGhterCofdartPoAththoAGhroAYdtheGUrdled
    earththeRroaPherCSelloYthePrePaUYCtheRhaFetheCtUllYorthUYtheUrhea
    rtCthehUllIUYdCUYtheUrFeUYCaYdtheGraYUteofYeIhaPSChUreUYtheUrPACW
    leCaYdtheUrTraUYCahUYtforWUSherCUHUCtheorderoffUFePodAloSUCCPall
  \end{Verbatim}

  \noindent
  Looking at the five-grams:

  \begin{Verbatim}
    "Ydthe": 1.0389610389610389
    "forth": 1.0389610389610389
    "orthe": 1.0389610389610389
    "theUr": 1.0389610389610389
    "Fearo": 0.7792207792207793
    "GUFea": 0.7792207792207793
    "PoAth": 0.7792207792207793
    "UFear": 0.7792207792207793
    "UYthe": 0.7792207792207793
    "YtheU": 0.7792207792207793
  \end{Verbatim}

  \noindent
  We see ``theUr'' occur frequently. This is most likely ``their''.
  Thus, we can replace `U' $\rightarrow$ `i'.

  \noindent
  Looking at the tri-grams

  \begin{Verbatim}
    "the": 4.651162790697675
    "ear": 1.550387596899225
    "rth": 1.550387596899225
    "Qor": 1.2919896640826873
    "Yth": 1.2919896640826873
    "art": 1.2919896640826873
    "ort": 1.2919896640826873
    "Ydt": 1.0335917312661498
    "aro": 1.0335917312661498
    "dth": 1.0335917312661498
  \end{Verbatim}

  We see ``Qor'' as one of the most-common 3-letter sequences.
  This is most-likely ``for''.

  \noindent
  Thus, we can swap `Q' $\rightarrow$ `f'.

  \begin{Verbatim}
    dearolddartPoAthGiFearoACefortheWolleGeoYthehillfortheloYeSiYeaTo
    FeheraYdtheloRaloYeCIholoFeherGiFearoACeGiFearoACeIithaIillforthe
    CoYCofolddartPoAthforthedaAGhterCofdartPoAththoAGhroAYdtheGirdled
    earththeRroaPherCSelloYthePrePaiYCtheRhaFetheCtillYorthiYtheirhea
    rtCthehillIiYdCiYtheirFeiYCaYdtheGraYiteofYeIhaPSChireiYtheirPACW
    leCaYdtheirTraiYCahiYtforWiSherCiHiCtheorderoffiFePodAloSiCCPall
  \end{Verbatim}

  \noindent
  We have the sequence ``WolleGe'', which could correspond to
  ``college''

  \noindent
  Thus, we can replace `W' $\rightarrow$ `c' and `G' $\rightarrow$ `g'.

  \begin{Verbatim}
    dearolddartPoAthgiFearoACeforthecollegeoYthehillfortheloYeSiYeaTo
    FeheraYdtheloRaloYeCIholoFehergiFearoACegiFearoACeIithaIillforthe
    CoYCofolddartPoAthforthedaAghterCofdartPoAththoAghroAYdthegirdled
    earththeRroaPherCSelloYthePrePaiYCtheRhaFetheCtillYorthiYtheirhea
    rtCthehillIiYdCiYtheirFeiYCaYdthegraYiteofYeIhaPSChireiYtheirPACc
    leCaYdtheirTraiYCahiYtforciSherCiHiCtheorderoffiFePodAloSiCCPall
  \end{Verbatim}

  \noindent
  We have `the college oY the hill'.
  `Y' is most likely `n'.

  \noindent
  We can also guess that, since we are talking about a college,
  ``dartPoAth'' should be ``dartmouth''.

  Thus, we can replace `Y' $rightarrow$ `n', `P' $\rightarrow$ `m', and `A' $rightarrow$ `u'.
  \begin{Verbatim}
    dearolddartmouthgiFearouCeforthecollegeonthehillfortheloneSineaTo
    FeherandtheloRaloneCIholoFehergiFearouCegiFearouCeIithaIillforthe
    ConCofolddartmouthforthedaughterCofdartmouththoughroundthegirdled
    earththeRroamherCSellonthemremainCtheRhaFetheCtillnorthintheirhea
    rtCthehillIindCintheirFeinCandthegraniteofneIhamSChireintheirmuCc
    leCandtheirTrainCahintforciSherCiHiCtheorderoffiFemoduloSiCCmall
  \end{Verbatim}

  \noindent
  We have the sequence ``dear old dartmouth giFe arouCe for the college on the hill...''

  \noindent
  We can guess that `F' should be `v' and `C' should be `s',
  so that we have ``dear old dartmouth give a rouse for the college on the hill''.

  \noindent
  We also have ``neI hamSshire'', which should likely be ``new hampshire''.
  We can replace `I' $\rightarrow$ `w' and `S' $\rightarrow$ `p'.

  \noindent
  We can also guess that ``aTove'' should be ``above'',
  and ``the loRal ones'' should be ``the loyal ones''.

  \noindent
  Finally, a number ``siH'' is probably ``six''.

  \begin{Verbatim}
    dearolddartmouthgivearouseforthecollegeonthehillforthelonepineabo
    veherandtheloyaloneswholovehergivearousegivearousewithawillforthe
    sonsofolddartmouthforthedaughtersofdartmouththoughroundthegirdled
    earththeyroamherspellonthemremainstheyhavethestillnorthintheirhea
    rtsthehillwindsintheirveinsandthegraniteofnewhampshireintheirmusc
    lesandtheirbrainsahintforciphersixistheorderoffivemodulopissmall
  \end{Verbatim}

  \newpage
  \noindent
  When spaced out, the above reads:

  \begin{Verbatim}
    dear old dartmouth, give a rouse
    for the college on the hill
    for the lone pine above her
    and the loyal ones who love her
    give a rouse, give a rouse with a will
    for the sons of old dartmouth
    for the daughters of dartmouth
    though round the girdled earth they roam
    her spell on them remains
    they have the still north in their hearts
    the hill winds in their veins
    and the granite of new hampshire in their muscles and their brains
    a hint for cipher six is the order of five modulo p is small.
  \end{Verbatim}
\end{Answer}
