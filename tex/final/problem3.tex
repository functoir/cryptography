\begin{problem}
  Affine cipher
  \begin{Verbatim}
    6917141364293641, 5044493105177484, 10208794241351887, 16394322558427148,
    11758121930809893, 15571898457877977, 7672722015089403, 13661070158473411,
    17999297470735005, 12313955920676335, 5960590266677512, 1613421779734456,
    1750819096862416, 3118598423638319, 14816640742963862, 4952241931583899,
    12257144082730227, 7862771476786858, 5006500927265261, 11323114722137903,
    22833602100630408, 8963415721169565, 15595638667025459, 8028339051359388,
    3385708046121353, 12190779082257523, 8983375210790796, 15571898457877977,
    15147654701575566, 16361132341028484, 5962327355151718, 8901193427034701,
    5179568152435730, 3672045789372412, 23610469115026974, 1577294047287513,
    15642317927380556, 15571898457877977, 10282634434851196, 10749617216933305,
    17838746455253440, 21499666401460178, 1037344909841996, 17413814796435480,
    16269186929768054, 10449344135634668, 24087490685235750, 10768725190149571,
    6484888204271905, 22185358129776042, 19377417029468988, 16267449841293848,
    16555381474675390, 21520574190817628, 14140526597210259, 19733309797334806,
    16283129124025650, 16538093542757166, 24098448719654436, 16798515250649044,
    13879801264995293, 10264930014031131, 7946076055449771, 18258106201941864,
    423054714981679, 17458353983971638, 9294184051519018, 19030921054252445
  \end{Verbatim}


  \noindent
  Using enhanced mind-reading techniques, Eve was able to able to extract the following
  information: Alice uses a general affine cipher, and the plaintext alphabet consists of blocks
  of five letters written as ASCII bytes (extended ASCII) and then interpreted as an integer
  modulo n. Apparently, even the coefficients of the affine cipher transformation are ASCII
  byte encodings of important codewords. (See strtoint and inttostr in final.sage for
  the precise encoding used.) The first part of the corresponding plaintext was also recovered:
  314077111660, 464400513312, 495875089509
  Unfortunately, Bob’s mind went dark and he could not disclose n.
\end{problem}

\begin{Answer}
  Since this an affine cipher, we know that $\exists$ $x, y \in \Z$ such that $c = p \cdot x + y$.

  \noindent
  First, we need to recover the base of the affine cipher, $n$.

  We know from the structure of the affine cipher that:

  \begin{align*}
    c_{i} &= p \cdot x_{i} + y \pmod n\\
    c_{i} &= p \cdot x_{i} + y - n_{i} \cdot n\\
    c_{1} &= p \cdot x_{1} + y - n_{1} \cdot n\\
    c_{2} &= p \cdot x_{2} + y - n_{2} \cdot n\\
    c_{3} &= p \cdot x_{3} + y - n_{3} \cdot n\\
    \\
    c_{1} - c_{2} &= p (x_{1} - x_{2}) - (n_{1} - n_{2}) \cdot n\\
    c_{2} - c_{3} &= p (x_{2} - x_{3}) - (n_{2} - n_{3}) \cdot n\\
    \\
    \text{To get rid of $p$ we calculate:}&&\\
    &(x_{2} - x_{3})(c_{1} - c_{2}) - (x_{1} - x_{2})(c_{2} - c_{3})&\\
  \end{align*}
  Giving:
  \begin{align*}
    ((x_{2} - x_{3})(n_{1}-n_{2}) - (m_{1}-m_{2})(n_{2}-n_{3}))p&= (x_{2} - x_{3})(c_{1} - c_{2}) - (x_{1} - x_{2})(c_{2} - c_{3})\\
  \end{align*}

  \noindent
  Meaning, $n$ is a factor of $(x_{2} - x_{3})(c_{1} - c_{2}) - (x_{1} - x_{2})(c_{2} - c_{3}) = -835256124266755612260628685$

  \noindent
  Taking the factors of the absolute value, we have:

  \noindent
  $835256124266755612260628685 = 5 \cdot 167051224853351122452125737$

  \noindent
  We can use $24610808569754243$ as out $n$, since it's the smallest value
  that is greater than all the ciphertext.

  \noindent
  Using the recovered plaintext, we can recover the coefficients of the affine cipher.

  \begin{align*}
    6917141364293641 &= 314077111660 \cdot x + y \pmod n \\
    5044493105177484 &= 464400513312 \cdot x + y \pmod n \\
    10208794241351887 &= 495875089509 \cdot x + y \pmod n
  \end{align*}

  \bigskip
  \centering
  \begin{tabular}{c | c | c |}
    & $6917141364293641$ & $= 314077111660 \cdot x + y \pmod n$ \\
    $-$ & $5044493105177484$ & $= 464400513312 \cdot x + y \pmod n$ \\
    \hline
    & $1872648259116157$ & $= -150323401652 \cdot x$\\
    \hline
    \hline
    & $1872648259116157$ & $\equiv 24610658246352591 \cdot x \pmod n$\\
    \hline
  \end{tabular}

  \flushleft{}
  Thus:
  \begin{align*}
    x &\equiv 1872648259116157 \cdot 24610658246352591^{-1} \pmod{24610808569754243} \\
    x &\equiv 1872648259116157 \cdot 2956863623047174 \pmod{24610808569754243} \\
    x &\equiv 289514745701 \pmod {24610808569754243} \\
    \\
    6917141364293641 &= 314077111660 \cdot 289514745701 + y \pmod n \\
    y &\equiv 5044493105177484 \cdot 289514745701 + y \pmod n \\
    y &\equiv 24610808569754212 \pmod n \\
    \\
    c &= 289514745701 \cdot p + 24610808569754212 \pmod{24610808569754243}\\
    p &= (c - 24610808569754212) \cdot 6618130068783420 \pmod{24610808569754243}\\
  \end{align*}

  \noindent
  Finally, we can decrypt the entire message:

  \begin{Verbatim}
    314077111660, 464400513312, 495875089509, 474400107552,
    147495347566,139476301088,444083740788,448378660128,
    457135256352,499967423520,521560989800,418598166626,
    435493412979,477284692585,465675313518,500036083828,
    491260571237,430040313714,189522408819,498760574317,
    435626861938,139391951220,139476301170,139358200172,
    139224637984,452903072872,418560108902,139476301088,
    418531057766,491327400992,521560989806,435492757620,
    477284954725,434336132973,435443164281,139140559717,
    198107482470,139476301088,448311747872,495790679328,
    482906432882,189523060837,472991231861,490170117986,
    139208106100,477284887920,478744506912,495790679394,
    478427029605,465792478752,138855082355,139208106094,
    478689651317,495869518112,444300616237,422540943459,
    418463906080,444016190766,146567877736,434333118057,
    444216713330,452823839604,435610744174,442925215602,
    139106807912,435706410016,452903062867,305716535328
  \end{Verbatim}

  by converting back to text, we get:
  \begin{Verbatim}
    I tell my students, ``When you get these jobs that 
    you have been so brilliantly trained for, just remember
    that your real job is that if you are free, you need to
    free somebody else. If you have some power, then your job
    is to empower somebody else. This is not just a grab-bag candy  game.''
    The Enigma ringstellung for cipher 4 is MSG.
  \end{Verbatim}

  \noindent
  When we decrypt the key, we get:
  \begin{Verbatim}
    Chloe Wofford
  \end{Verbatim}
  
\end{Answer}
