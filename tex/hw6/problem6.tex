
\begin{problem}
Pollard's $p-1$ factorizing algorithm uses the following idea.  Let
$p$ be a prime divisor of an integer $n$ and let $d$ be a divisor of
$p-1$.  Suppose that an integer $a$ has multiplicative order $d$ in
$\Z/p\Z$, i.e., that $a^d \equiv 1 \bmod p$.  If
$a^d \not\equiv 1 \bmod n$ then $\gcd(a^d-1,n)$ is a proper divisor of
$n$.  This can be turned into an algorithm: choose a random $1 < a
< n$ and inductively set $a_1=a$ and $a_i \equiv a_{i-1}^i \bmod n$ for
$i > 1$, and check whether $\gcd(a_i-1,n) \neq 1,n$.  If you ever
get $\gcd(a_i-1,n)=n$, then halt, choose a different $a$, and start over.
\begin{enumerate} \renewcommand{\itemsep}{3mm}
\item Find a nontrivial factor of $n=47371$ using this algorithm,
starting with $a=2$.  Show what happens at each step.
\end{enumerate}

\begin{Answer}
  I wrote a simple implementation of this algorithm.
  \color{zaffre}
  \begin{Verbatim}[commandchars=\\\{\}]
pollard' :: Integer -> IO (Integer, Integer)
pollard' n = iter n 2 1
  where
    iter n a i = do
      let g = gcd (a - 1) n
      printf "n = %5d   a_%d = %5d   gcd = %5d" n i a g

      \crim{if g /= 1 && g /= n then}
      \crim{  return (g, n div g)}
      \crim{else if g == n then}
      \crim{  iter n (randomInt 2 n) 1}
      \crim{else}
      \crim{  iter n ((a ^ i) mod n) (i + 1)}
      \color{black}
  \end{Verbatim}
  \color{black}

  \noindent
  Results:

  \begin{Verbatim}
    ghci> pollard' 47371
    n = 47371       a_1 =     2     gcd =     1
    n = 47371       a_2 =     4     gcd =     1
    n = 47371       a_3 =    64     gcd =     1
    n = 47371       a_4 =  7882     gcd =     1
    n = 47371       a_5 = 34800     gcd =     1
    n = 47371       a_6 = 31941     gcd =     1
    n = 47371       a_7 = 15368     gcd =   127
    (127,373)
  \end{Verbatim}

  Thus, we can see that $127$ and $373$ are non-trivial factors of $47371$.


\end{Answer}

\begin{enumerate}
\setcounter{enumi}{1}
\item Show that $a_i \equiv a^{i!} \bmod n$ for all $i \geq 1$.

\begin{Answer}
  We can prove this by induction on $i$:

  \noindent
  For the base case, we have the definition of $a_1$:
    \[a_{1} \equiv a \equiv a^{1!} \bmod n\]

  \noindent
  In the inductive step, we have the definition that, for all $1 < i < n$,
  $a_i \equiv a_{i-1}^i \bmod n$.

  \noindent
  Thus, we have the base-case that $a_{1} = a^{1!} \pmod n$,
  and, for every next $i$,
  
  \noindent
  $a_i \equiv a_{i-1}^{i} \pmod n = a^{{i \cdot (i-1)}!} \equiv a^{i!} \pmod n$.

\end{Answer}

\newpage
\item If $p-1 | N$ for some integer $N$, show that $p | a^N-1$ for any
integer $a$ relatively prime to $n$.

\begin{Answer}
    \begin{align*}
      p - 1 | N &\Rightarrow \exists k \in Z_{+} \backepsilon N = k \cdot (p -1) \\
      a^N = a^{k \cdot (p -1)} &\equiv {(a^{p-1})}^{k} \pmod p \\
      \However a ^N = a^{p-1} &\equiv 1 \pmod p \text{ per Fermat's Little Theorem} \\
      \Therefore a^N = {(a^{p-1})}^{k} &\equiv 1^k \equiv 1 \pmod p \\
      \Therefore a^N - 1 &\equiv 0 \pmod p \\
      \Therefore p &\;|\; (a^N - 1)
    \end{align*}
\end{Answer}

\item Explain why this algorithm runs quickly if there is a reasonably
small $i \geq 1$ such that $(p-1) | {i!}$, and  deduce that in this case,
$p-1$ must be quite smooth. 

\begin{Answer}
  The algorithm works because of Fermat's Little Theorem and the property
  it induces---that if $(p-1) | {i!}$ then $p | (a^{i!} - 1)$ for any $a$.

\end{Answer}

\item A prime number $p$ is called a \textbf{safe prime} if $p-1 =
2p'$ for a prime number $p'$.  Explain why the existence of Pollard's
$p-1$ algorithm implies that for the modulus $n=pq$ in RSA, both $p$
and $q$ should be chosen to be safe primes.
\end{enumerate}

\medskip

A prime number $p'$ such that $2p'+1$ is also prime is called a
\textbf{Sophie Germain prime}.  So $p$ is safe if and only if
$(p-1)/2$ is a Sophie Germain prime.  Sophie Germain was a late
18th/early 19th century French mathematician who, due to prejudices of
the time, was not allowed to attend lectures at the \'Ecole
Polytechnique in Paris.  So she obtained the lecture notes and started
corresponding with famous mathematicians Lagrance and Gauss about her
ideas under a male pseudonym.  She did fantastic research work in
number theory, most famously, she presented a strategy to prove
Fermat's Last Theorem for prime exponents related to Sophie Germain
primes, which is how her primes were named.  She also did important
work in philosophy and in the theory of elasticity, for which she was
the first woman to win a prize from the Paris Academy of Sciences, in
1816.  You can read more about her history and mathematics in
Wikipedia.
\end{problem}
