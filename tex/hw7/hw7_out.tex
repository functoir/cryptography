\documentclass[11pt]{amsart}
\usepackage{amssymb,amsthm,amsmath,amstext}
\usepackage{mathdots}

\usepackage[left=1.2in,right=1.2in,top=1.0in,bottom=1.2in,nohead,nofoot]{geometry}

\newcommand{\Z}{\mathbb{Z}}
\newcommand{\Q}{\mathbb{Q}}
\newcommand{\F}{\mathbb{F}}

\pagestyle{empty}

\newcounter{problem}
\setcounter{problem}{0}

\newenvironment{problem}[1][]%
{%
\stepcounter{problem} \vspace{.2cm} \noindent {\bf \arabic{problem}.} {\it #1}~ %
}{%
\vspace{.2cm}%
} 

\renewcommand{\theenumi}{\alph{enumi}}

\begin{document}

\thispagestyle{empty}

\title{} 

%\date{\today}

\noindent{} \sc
Dartmouth College Department of Mathematics \\ \bf
Math 75 Cryptography \\ \rm
Spring 2022\\[2ex]
Problem Set \# 7 (upload to Canvas by Tuesday, May 24, 12:00 pm EDT)

\bigskip

\noindent {\bf Problems:}

\begin{problem}
For the following integers either provide a witness for the
compositeness of $n$ or conclude that $n$ is probably prime by
providing $5$ numbers that are not witnesses.  Recall that a witness
for the compositeness of $n$ is an integer $a \in \Z$ such that, if we
write $n-1 = 2^k u$, where $u$ is odd, then $a$ satisfies $a
\not\equiv 0 \pmod n$ and  $a^{u} \not\equiv 1 \pmod n$
 and $a^{2^i u} \not\equiv -1 \pmod n$ for all $i = 1, \dotsc, k-1$.
\begin{enumerate}\renewcommand{\itemsep}{3mm}
\item $n=1009$.
\item $n=2009$.
\end{enumerate}
\end{problem}


\begin{problem}
Using big-$O$ notation, estimate the number of bit operations required to perform the witness test on $n \in \Z_{>0}$ enough times so that, if $n$ passes all of the tests, it has less than a $10^{-m}$ chance of being composite.
\end{problem}


\begin{problem}
Factor $53477$ using the Pollard rho algorithm.
\end{problem}


% \begin{problem}
% Suppose that $n$ balls are randomly thrown into $m$ bins.
% \begin{enumerate}\renewcommand{\itemsep}{3mm}
% \item Approximately what is the probability that there is a bin with two balls in it, assuming $m$ is much larger than $n$?
% \item What is the probability that there is a bin with no balls in it?  If $m$ is large, and $n$ is large in comparison to $m$, show that this probability is approximately $me^{-n/m}$.  \emph{[Hint: Use $\lim_{x \to \infty} (1+1/x)^x = e$.]}  

% \item In terms of $m$, what is the smallest value of $n$ so that there is a $\geq 1/2$ chance that no bin is empty?  

% \item Suppose you are asked to fill an auditorium with people so that every day is a birthday for some person in the auditorium.  What is the smallest number of people which gives you better than an even chance?
% \end{enumerate}
% \end{problem}


\begin{problem}
Fermat and sieving.
\begin{enumerate}\renewcommand{\itemsep}{3mm}
\item Find three nontrivial factors of
$n=999999999999999999999999999999999919$ by hand.

\item Let $n=2^{29}-1$.  Given that
\begin{align*} 
258883717^2 &\equiv -2\cdot 3 \cdot 5 \cdot 29^2  {} \pmod n \\
301036180^2 &\equiv -3\cdot 5 \cdot 11 \cdot 79   {} \pmod{n} \\
126641959^2 &\equiv 2 \cdot 3^2 \cdot 11 \cdot 79 {~} \pmod n
\end{align*}
discover a factor of $n$.
\end{enumerate}
\end{problem}


\begin{problem}
Discrete logarithms.
\begin{enumerate}\renewcommand{\itemsep}{3mm}
\item Let $p=101$.  Compute $\log_2 11$ (using complete enumeration by hand).

\item Let $p=27781703927$ and $g=5$.  Suppose Alice and Bob engage in
a Diffie-Hellman key exhange; Alice chooses the secret key
$a=1002883876$ and Bob chooses $b=21790753397$.  Describe the key
exchange: what do Alice and Bob exchange, and what is their common
(secret) key?  \emph{[You may want to use a computer!]}

\item Let $p = 1021$.  Compute $\log_{10} 228$ using the baby step-giant
step algorithm. Show the output of, and explain all steps in, your computation.

\item Let $p = 1801$.  Compute $\log_{11} 249$ using the
Pohlig--Hellman algorithm.  Show the output of, and explain all steps
in, your computation.  You'll want to remind yourself of how to solve
systems of congruence equations using Sunzi's theorem: To find $x \in
\Z$ satisfying $x \equiv a_i \pmod {n_i}$ for $i=1,\dotsc,k$, first
define integers $N_i = \prod_{j\neq i} n_i$ and $M_i \equiv N_i^{-1}
\pmod {n_i}$ for all $i = 1,\dotsc,k$, and then $x = \sum_{i=1}^k a_i
N_i M_i$ works.
\end{enumerate}
\end{problem}


% \begin{problem}
% \end{problem}


% \begin{problem}
% \end{problem}

% \begin{problem}
% In a modified Diffie-Hellman key exchange protocol, Alice and Bob choose a large prime $p$ which they make public, but when they choose a primitive root $g$ for $p$ they decide for safety to keep it secret.  Alice sends $x \equiv g^a \pmod{p}$ to Bob and Bob sends $y \equiv g^b \pmod{p}$ to Alice.  Suppose Eve bribes Bob to tell her the values of $b$ and $y$.  Suppose that $\gcd(b,p-1)=1$.  Show how Eve can determine $g$ from the knowledge of $p,y$ and $b$.
% \end{problem}


\vfill

\end{document}





