\documentclass[11pt]{amsart}

\usepackage[left=1.2in,right=1.2in,top=1.5in,bottom=1.5in,nohead,nofoot]{geometry}

\input{../common/macros.tex}

\pagestyle{empty}

\newcounter{problem}
\setcounter{problem}{0}

\renewcommand{\theenumi}{\alph{enumi}}

\begin{document}

\thispagestyle{empty}

\title{} 

\date{\today}

\noindent{\sc
Dartmouth College Department of Mathematics \\ \bf
Math 75 Cryptography \\ \rm
Spring 2022\\[2ex]
Problem Set \# 3 (upload to Canvas by Friday, April 22, 10:10 am EDT)}

\bigskip

\noindent {\bf Problems:}


\begin{problem}
Consider the affine cipher with $\calP=\calC=\Z/n\Z$.  
\begin{enumerate}\renewcommand{\itemsep}{3mm}
\item Suppose $n=541$ and we take the key $(a,b)=(34,71)$.  Encrypt the plaintext $m=204$, and decrypt the ciphertext $c=431$.
\item Eve intercepts a ciphertext from Alice and through espionage she learns that the letter $x \in \calP$ is encrypted as $y \in \calC$ in this message.  Show that Eve can decrypt the message using $O(n)$ trials.  
\item Now suppose that (contrary to Kerckhoffs's principle) the integer $n$ is not public knowledge.  Is the affine cipher still vulnerable if Eve manages to steal a plaintext/ciphertext pair?  How might Eve break the system?
\end{enumerate}
\end{problem}

\begin{problem}
Encrypt the message 
\begin{center}
\textsf{Why is a raven like a writing desk} 
\end{center}
using the Vigen\`ere cipher with keyword \texttt{rabbithole}.
\end{problem}

\begin{problem}
Decrypt the following message, which was encrypted using a Vigen\`ere cipher.
\begin{verbatim}
          mgodt beida psgls akowu hxukc iawlr csoyh prtrt udrqh cengx
          uuqtu habxw dgkie ktsnp sekld zlvnh wefss glzrn peaoy lbyig
          uaafv eqgjo ewabz saawl rzjpv feyky gylwu btlyd kroec bpfvt
          psgki puxfb uxfuq cvymy okagl sactt uwlrx psgiy ytpsf rjfuw
          igxhr oyazd rakce dxeyr pdobr buehr uwcue ekfic zehrq ijezr
          xsyor tcylf egcy
\end{verbatim}
\begin{enumerate}\renewcommand{\itemsep}{3mm}
\item Use the method of displacement coincidences to guess the key length.
\item Use the Kasiski test to give more evidence for your guess for the key length.  
\item Use frequency analysis with the guessed key length to decrypt the message.
\end{enumerate}
\emph{[You are encouraged to use a computer.]}
\end{problem}

\begin{problem}
Consider the quadratic map
\begin{align*}
E:\Z/n\Z &\to \Z/n\Z \\
x &\mapsto x^2+ax+b
\end{align*}
with $a,b \in \Z/n\Z$.  Show that if $n \neq 2$, then $E$ is
\emph{never} an encryption function (i.e., $E$ cannot be inverted).
What can you say about other maps $x \mapsto f(x)$ where $f(x) \in
\Z[x]$, in particular, are any polynomial maps of higher degree invertible?
\end{problem}

\begin{problem}
Let $D_n = \{x \in \R^n : \sum_{i=1}^n x_i^2 = 1\}$ be the unit sphere
in $\R^n$.  Fix $x \in D_n$ and consider the function $\psi_x : D_n
\to \R$ defined by
\[
\psi_x(y) = x \cdot y = \sum_{i=1}^n x_i y_i.
\]
Show that the function $\psi_x$ achieves a unique maximum at $x=y$.
How does this relate to frequency analysis?
\end{problem}

\medskip

\noindent {\bf Challenge problem:} (Try it for fun, you are not
required to submit written-up solutions, unless you are a graduate
student enrolled in the class.)

\begin{problem}
Let $n,k \in \Z_{>0}$ and recall the general linear group $\GL_k(\Z/n\Z)$.
\begin{enumerate}\setlength{\itemsep}{3mm}
\item Write down all the elements of $\GL_2(\Z/2\Z)$.  What more
commonly known group is this isomorphic to?

\item If $n=p$ is a prime number, prove that $\GL_k(\Z/p\Z)$ has
$(p^k-1)(p^k-p)\dotsm (p^k-p^{k-1})$ elements.  \emph{[Use linear
algebra over the field $\Z/p\Z$ and think of building your matrix one
column at a time.]}

\item Prove that if $n,m$ are relatively prime positive integers, then
\[
\# \GL_k(\Z/nm\Z) = \# \GL_k(\Z/n\Z) \cdot \# \GL_k(\Z/m\Z).  
\]
The following subparts will provide a guide to an algebraic proof of
this fact (not all of these require a proof, they are a kind of series
of hints to guide your work).

\begin{enumerate}\setlength{\itemsep}{3mm}
\item For $n,m$ relatively prime, the map $\phi : \Z/nm\Z \to \Z/n\Z
\times \Z/m\Z$, defined by $a \mapsto (a \bmod n, a \bmod m)$, is an isomorphism of groups.  We can
write $\phi(a) = (\phi_n(a),\phi_m(a))$ where $\phi_n : \Z/nm\Z \to
\Z/n\Z$ is the reduction modulo $n$ homomorphism and similarly for
$\phi_m$.  In fact, $\phi$ is an isomorphism of rings with 1, i.e.,
respects multiplication and the multiplicative identity.

  \item Promote $\phi$ to an isomorphism $\Phi : M_k(\Z/nm\Z) \to
    M_k(\Z/n\Z) \times M_k(\Z/m\Z)$ of rings with 1 by sending a
    matrix $A = (a_{ij})_{1 \leq i,j \leq k}$ to the
    pair $(\Phi_n(A), \Phi_m(A))$, where $\Phi_n(A) =
    (\phi_n(a_{ij}))_{1\leq i,j\leq k}$ is the result of reducing all
    entries of $A$ modulo $n$, and similarly for $\Phi_m(A)$.  First
    you have to prove that $\Phi$ is a ring homomorphism, then that it
    is injective and surjective, which relies crucially on the
    injectivity and surjectivity of $\phi$.

\item Prove that $\phi(\det(A)) = (\det(\Phi_n(A)), \det(\Phi_m(A)))$ for all $A \in M_k(\Z/nm\Z)$.  Colloquially, this says that
  $\phi$ and $\Phi$ ``respect'' the determinant.

\item Prove that $A \in M_k(\Z/nm\Z)$ is invertible if and only if
  $\Phi(A)$ is an invertible element of the ring $M_k(\Z/n\Z) \times M_k(\Z/m\Z)$ if and
  only if both $\Phi_n(A) \in M_k(\Z/n\Z)$ and $\Phi_m(A) \in
  M_k(\Z/m\Z)$ are invertible.  Conclude that $\Phi$ induces a group
  isomorphism $\GL_k(\Z/nm\Z) \crim{\isom} \GL_k(\Z/n\Z) \times
  \GL_k(\Z/m\Z)$ and as a consequence, we get the desired formula.
\end{enumerate}

\item Recall the affine cipher with $\calP = \calC = (\Z/n\Z)^k$ and
with key $A \in \GL_k(\Z/n\Z)$.  If Eve discovers the encryption of
$k$ plaintext elements, prove that the probability that she can solve
for the key is $\#\GL_k(\Z/n\Z)/n^{k^2}$.  Compute this probability
for $n=26$ and $k=2,3,4$.  \emph{[This was done a bit too quickly in
lecture, so check it yourself.]}

\item After experimenting, what can you say about this probability as $k \to \infty$ or as
$n\to \infty$?
\end{enumerate}
\end{problem}

\vfill

\end{document}

