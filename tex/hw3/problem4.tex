
\begin{problem}
Consider the quadratic map
\begin{align*}
E:\Z/n\Z &\to \Z/n\Z \\
x &\mapsto x^2+ax+b
\end{align*}
with $a,b \in \Z/n\Z$.  Show that if $n \neq 2$, then $E$ is
\emph{never} an encryption function (i.e., $E$ cannot be inverted).
What can you say about other maps $x \mapsto f(x)$ where $f(x) \in
\Z[x]$, in particular, are any polynomial maps of higher degree invertible?
\end{problem}
\begin{Answer}
An encryption function has to be invertible (for decryption),
and invertible functions must be injections and a surjections (i.e. bijections).
These conditions are not satisfied by the quadratic map
$E:\Z/n\Z \to \Z/n\Z \backepsilon x \mapsto x^2+ax+b$
unless we restrict its domain and range to $\Z/2\Z$.
First, for a multiplicative map over $\Z/n\Z$ to be invertible, $\Z/n\Z$ must be a field --- otherwise some elements
in $\Z/n\Z$ will not have multiplicative inverses.

\noindent
Consider some finite field $\calF_n$ formed by taking the integers modulo $n$, where $n$ is a prime number.
As an example, we'll take $\calF_5$

\noindent
Every element in $\calF$ except $0$ has a multiplicative inverse and an additive inverse.
For example:
\noindent
Let's define $\calF_5 = \{0,1,2,3,4\}$
\noindent
Then, we can define multiplicative inverses and additive inverses in $\calF_5$ as:

% \begin{table}[h!]
\centering
\begin{tabular}{||c | c | c||} 
  \hline
  $f_i$ & $f_i^{-1}$ & $-f_i$\\ [0.5ex] 
  \hline\hline
  1 & 1 & 4\\ 
  2 & 3 & 3\\
  3 & 2 & 2\\
  4 & 4 & 1\\[1ex] 
  \hline
\end{tabular}
% \caption{Table to test captions and labels.}
\label{table:1}
% \end{table}

\justifying
\noindent
As in all number systems (that I know of\ldots I'd like to know of exceptions),
$f_i^{2} = (f_i^{-1})^{2}$.

\noindent
Similarly, $f_i + (-f_i) = 0$

\noindent
Examples in $F_5$:
\begin{align*}
  3^2 &\equiv 9 \pmod 5 \equiv 4 \equiv 2^2\\
  3 + 2 &\equiv 5 \pmod 5 \equiv 0\\
\end{align*}

\noindent
Hence\ldots the quadratic map \zaff{$E:\Z/n\Z \to \Z/n\Z \backepsilon x \mapsto x^2+ax+b$}
is guaranteed to not be surjective under these conditions:

\begin{enumerate}
\item If $a = 0$, then $E(x^{-1}) \equiv x^2 + b \equiv E(x)$
\item Generally, $E(-a) \equiv a^{2} - a^{2} + b \equiv b \equiv E(0)$ 
\end{enumerate}

\bigskip
\noindent
\color{zaffre}
However, if we limit the domain and range to $\calF_2$, then the only non-zero element is $1$, and:
\begin{enumerate}
  \item If $a = 0$, then $E(x^{-1}) \equiv x^2 + b \equiv E(x)$, \crim{but $1 \equiv 1^{-1} \pmod 2$}
  \item \crim{$-1 \equiv 1 \pmod 2$}, and $E(-1) = E(1) = 1 + a + b \nequiv E(0)$ 
\end{enumerate}
\color{black}
\bigskip
\noindent
In general, it is possible for higher-order polynomials to be invertible, but that property does not hold for \emph{all}
higher-order polynomials. Careful thought should be put in choosing a polynomial as an encryption function (or, just use modern
techniques?).

\noindent
For instance, $x \mapsto x^3$ (and any such function as $x \mapsto x^{\text{odd power}}$, it seems)
is invertible in any finite field, but $x \mapsto x^4$ (and any other $x \mapsto x^{\text{even power}}$ function) is not.
\end{Answer}
