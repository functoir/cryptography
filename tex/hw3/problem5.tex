\begin{problem}
Let $D_n = \{x \in \R^n : \sum_{i=1}^n x_i^2 = 1\}$ be the unit sphere
in $\R^n$.  Fix $x \in D_n$ and consider the function $\psi_x : D_n
\to \R$ defined by
\[
\psi_x(y) = x \cdot y = \sum_{i=1}^n x_i y_i.
\]
Show that the function $\psi_x$ achieves a unique maximum at $x=y$.
How does this relate to frequency analysis?
\end{problem}

\begin{Answer}
\text{Let: }
\begin{align*}
  \vec{x} &= \langle x_1, x_2, \dots, x_n\rangle \backepsilon \sum_{i=1}^n x_i^2 = 1 \zaff{\Rightarrow \left | \vec{x} \right | = 1}\\
  \vec{y} &= \langle y_1, y_2, \dots, y_n\rangle \backepsilon \sum_{i=1}^n y_i^2 = 1 \zaff{\Rightarrow \left | \vec{y} \right | = 1}\\
  \vec{x} \cdot \vec{y} &=\left | \vec{x} \right |\left | \vec{y} \right |\cos\theta
  \le \left | \vec{x} \right |\left | \vec{y} \right | \crim{\text{    (since $\cos \theta \le 1$)}}
\end{align*}
\noindent
To achieve a maximum:
\begin{align*}
  \underbrace{\cos \theta \le 1}_{\text{maximize this}} &\Rightarrow \cos \theta = 1\\
  \\
  \angle(\vec{x}, \vec{y}) &= \arccos 1 = 0
\end{align*}

\noindent
Thus, we \emph{know} that $\vec{x}$ and $\vec{y}$ are the same,
since $ \angle(\vec{x}, \vec{y}) = 0$,
and $\left | \vec{x} \right | = \left | \vec{y} \right| = 1$
\crim{(the vectors have the same direction and the same magnitude)}.
\newline
\newline
\noindent
\color{zaffre}
Suppose we have $\vec{f} = \langle f_1, f_2, \dots, f_n\rangle$
corresponding to the frequencies of the
$n$ letters in an alphabet, such that $\sum_{i=1}^n f_i = 1$.
Then, the Cauchy-Schwartz inequality implies that $\vec{f} \cdot \vec{f}$
gives a greater value than $\vec{f} \cdot \vec{f}_{\to k}$ where $\vec{f}_{\to k}$ is the vector
obtained by shifting and cycling the elements in $\vec{f}$ to the right by $k$ positions.
\newline
On the vigen\`ere cipher, for instance, this means that the frequencies (or coincidences, an approximation for frequencies)
will be maximized when the shift matches the key-length.
\end{Answer}
