\begin{problem}

Encrypt the message 
\begin{center}
\textsf{Why is a raven like a writing desk} 
\end{center}
using the Vigen\`ere cipher with keyword \texttt{rabbithole}.
\end{problem}

\begin{Answer}

The encryption is \Verb#``NHZJATYOGIELJLMTDFTXZNHEMLR''#

\bigskip
\noindent
\begin{center}\textbf{Algorithm}\end{center}

\noindent
I wrote a program to encrypt and decrypt per the Vigen\`ere cipher.

\color{crimson}
\begin{Verbatim}[commandchars=\\\{\}]
  -- | Get the "vigenere complement" of a character.
  --
  -- The complement of 'A' is itself (shift by 0),
  --
  -- the complement of 'B' is 'Z' (shift by 1 and -1), etc.
  \zaff{invChar :: Char -> Char}
  \zaff{invChar char = chr (ord 'Z' - (charToInt char - 1))}
\end{Verbatim}
\color{black}
\noindent
For convenience, we can define a function that maps invChar over a word:
\color{crimson}
\begin{Verbatim}[commandchars=\\\{\}]
  -- | Get the "vigenere complement" of a word.
  --
  -- maps the complement of each character in the word.
  \zaff{invWord :: String -> String}
  \zaff{invWord = map invChar}
\end{Verbatim}
\color{black}

\noindent
Also for convenience, I wrote a function that repeats any sequence
infinitely many times. This creates an infinite sequence,
but since Haskell is a lazy language we can ``take''
the first $n$ elements out of such a sequence.

\color{crimson}
\begin{Verbatim}[commandchars=\\\{\}]
  -- | Repeat a sequence infinitely many times.
  --
  -- This is a lazy function, so it will not evaluate the
  -- sequence infinitely many times.
  \zaff{repeat :: [a] -> [a]}
  \zaff{repeat seq = seq ++ repeat seq}
\end{Verbatim}

\noindent
\black{Finally, we can write our encryption function:}

\begin{Verbatim}[commandchars=\\\{\}]
  -- | Encrypt a word using the Vigen\`ere cipher.
  -- NOTE: `zipWith' is a builtin function that takes a function 
  -- and two sequences and applies the function on
  -- corresponding elements in the sequences to generate a new sequence.
  \zaff{encrypt :: String -> String -> String}
  \zaff{encrypt text keyword = zipWith shiftChar cleanedText repeatedKeyword}
    where
      \zaff{cleanedText = clean text}                    \crim{-- drops spaces and punctuation}
      \zaff{repeatedKeyword = take n (repeat keyword)}   \crim{-- gets first n letters in sequence}
      \zaff{n = length cleanedText}
\end{Verbatim}
\color{black}
\noindent
And we can define decryption as encryption with the inverse of the key, i.e.
the respective letters that undo the shifts done during encryption:

\color{crimson}
\begin{Verbatim}[commandchars=\\\{\}]
  -- | Decrypt a word using the Vigen\`ere cipher.
  --
  -- We do the equivalent of encryption with the Vige\`ere `inverse' of the keyword.
  \zaff{decrypt :: String -> String -> String}
  \zaff{decrypt text keyword = encrypt text (invWord keyword)}
\end{Verbatim}
\color{black}

% \newpage
\begin{center}\textbf{Results}\end{center}

\color{crimson}
\begin{Verbatim}[commandchars=\\\{\}]
$ encrypt \zaff{``Why is a raven like a writing desk''} ``rabbithole''
\teal{``NHZJATYOGIELJLMTDFTXZNHEMLR''}


$ decrypt \teal{``NHZJATYOGIELJLMTDFTXZNHEMLR''} ``rabbithole''
\zaff{``WHYISARAVENLIKEAWRITINGDESK''}
\end{Verbatim}
\color{black}
\end{Answer}
