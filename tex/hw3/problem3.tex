
\begin{problem}
Decrypt the following message, which was encrypted using a Vigen\`ere cipher.
\begin{Verbatim}
      mgodt beida psgls akowu hxukc iawlr csoyh prtrt udrqh cengx
      uuqtu habxw dgkie ktsnp sekld zlvnh wefss glzrn peaoy lbyig
      uaafv eqgjo ewabz saawl rzjpv feyky gylwu btlyd kroec bpfvt
      psgki puxfb uxfuq cvymy okagl sactt uwlrx psgiy ytpsf rjfuw
      igxhr oyazd rakce dxeyr pdobr buehr uwcue ekfic zehrq ijezr
      xsyor tcylf egcy
\end{Verbatim}

\begin{enumerate}\renewcommand{\itemsep}{3mm}
  \item Use the method of displacement coincidences to guess the key length.
  \item Use the Kasiski test to give more evidence for your guess for the key length.  
  \item Use frequency analysis with the guessed key length to decrypt the message.
\end{enumerate}
\emph{[You are encouraged to use a computer.]}
\end{problem}
% \newpage
\noindent
\begin{Answer}
\begin{center}\textsc{Key Length Estimation}\end{center}
\bigskip
After counting displacement coincidences, I found $7$ has the highest number of coincidences.

\color{zaffre}
\begin{Verbatim}[commandchars=\\\{\}]
  1: 7
  2: 6
  3: 11
  4: 11
  5: 9
  6: 11
  \crim{7: 15}
  8: 4
  9: 10
  10: 12
  11: 11
  12: 9
  13: 12
  \crim{14: 17    -- could this be because it is a multiple of 7?}
  15: 10
  16: 6
  17: 11
  18: 11
  19: 7
\end{Verbatim}
\color{black}
% \pagebreak
\begin{center}\textsc{Kasiski Test}\end{center}

I wrote a program that analyzes the recurrences of $n$-grams in the text.
\noindent
\color{zaffre}
\begin{Verbatim}[commandchars=\\\{\}]

  --- 3-grams
  *VigenereCipher> run 3
  awl: [26,117]       \crim{difference: 91}
  ehr: [227,241]      \crim{difference: 14}
  gki: [61,152]       \crim{difference: 91}
  gls: [12,173]       \crim{difference: 161}
  lsa: [13,174]       \crim{difference: 161}
  psg: [10,150,185]   \crim{difference: [140, 175, 35]}
  sgl: [11,84]        \crim{difference: 73}
  tps: [149,191]      \crim{difference: 42}
  uxf: [156,160]      \crim{difference: 4}
  wlr: [27,118,181]   \crim{difference: [91, 154, 63]}

  --- 4-grams
  *VigenereCipher> run 4
  awlr: [26,117]        \crim{difference: 91}
  glsa: [12,173]        \crim{difference: 161}
\end{Verbatim}
\color{black}
\noindent
With a length of 3, we see that several $n$-grams recur in the encrypted message.
\newline
Per the \textbf{Kasiski Test}, most of the differences in position
of repeated $n$-grams should be multiples of the key-length ($7$ in this case).
We see that $\left\{14, 35, 42, 63, 91, 140, 154, 161, 175\right\}$ are all multiples of $7$.
Only $\{4, 73\}$ are not multiples of $7$.
% \pagebreak
\begin{center}
\textsc{Frequency Analysis}
\end{center}

\noindent
Looking at the highest frequencies over each zeroth, first,
second, third, fourth, fifth, and sixth letter modulo $7$:
\begin{multicols*}{2}
\begin{Verbatim}[commandchars=\\\{\}]
1   [ 'i': 15.789473684210526,
      'e': 10.526315789473685
      's': 10.526315789473685
      'l': 7.894736842105263
      'r': 7.894736842105263
      ...
    ],
2   [ 'r': 15.789473684210526
      'a': 10.526315789473685
      'y': 10.526315789473685
      'b': 7.894736842105263
      'e': 7.894736842105263 
      ...
    ],
3   [ 'u': 18.42105263157895
      'k': 15.789473684210526
      'o': 13.157894736842104
      't': 7.894736842105263
      'z': 7.894736842105263
      ...
    ],
4   [ 'p': 13.157894736842104
      'c': 10.526315789473685
      'e': 10.526315789473685
      't': 10.526315789473685
      'w': 7.894736842105263
      ...
    ],
5   [ 's': 13.157894736842104
      'a': 10.526315789473685
      'e': 10.526315789473685
      't': 10.526315789473685
      'c': 7.894736842105263
      ...
    ],
6   [ 'g': 16.216216216216218
      'b': 10.81081081081081
      'u': 10.81081081081081
      'y': 10.81081081081081
      'f': 8.108108108108109
      ...
    ],
7   [ 'l': 13.513513513513514
      'w': 10.81081081081081
      'd': 8.108108108108109
      'h': 8.108108108108109
      'p': 8.108108108108109
      'f': 5.405405405405405
      ...
    ]
\end{Verbatim}
\end{multicols*}
\noindent
\noindent
Suppose the keyword $k$ = $k_{1}k_{2}k_{3}k_{4}k_{5}k_{6}k_{7}$
where $k_{1}$ is the first letter of the message, etc.
\newline
Since we expect the most common letters to have similar recurrence across the text,
we can pick out one recurring frequency ($15.789473684210526$) and check the values closest to that frequency.
\newline
We can expect that:

\begin{align*}
\exists c_{i} \in \calC, \backepsilon c_{i} \begin{cases}
  \underset{k_{1}}{\Rightarrow} \text{`i'}\\
  \underset{k_{2}}{\Rightarrow} \text{`r'}\\
  \underset{k_{3}}{\Rightarrow} \text{`k'}\\
  \underset{k_{4}}{\Rightarrow} \text{`p'}\\
  \underset{k_{5}}{\Rightarrow} \text{`s'}\\
  \underset{k_{6}}{\Rightarrow} \text{`g'}\\
  \underset{k_{7}}{\Rightarrow} \text{`l'}
\end{cases} 
\end{align*}

From the above, we can guess that:
\begin{align*}
  \mathbf{let\ } k_{1} &\equiv \text{`A'} \\
  k_{2} - k_{1} = \text{`r'} - \text{`i'} = 9 \Rightarrow k_{2} &\equiv \text{`J'}\\
  k_3 - k_1 = \text{`k'} - \text{`i'} = 2 \Rightarrow k_{3} &\equiv \text{`C'}\\
  k_4 - k_1 = \text{`p'} - \text{`i'} = 7 \Rightarrow k_{4} &\equiv \text{`H'}\\
  k_5 - k_1 = \text{`s'} - \text{`i'} = 22 \Rightarrow k_{5} &\equiv \text{`W'}\\
  k_6 - k_1 = \text{`g'} - \text{`i'} = 24 \Rightarrow k_{6} &\equiv \text{`Y'}\\
  k_7 - k_1 = \text{`l'} - \text{`i'} = 3 \Rightarrow k_{7} &\equiv \text{`D'}\\
\end{align*}

We can now run a brute-force shift cipher attack on the relations of the keyword,
and try to look for a recurring pattern.
\begin{multicols*}{2}
\begin{Verbatim}[commandchars=\\\{\}]
  *ShiftCipher> bruteforce ``AJCHWYD''
  0: ajchwyd
  1: zibgvxc
  2: yhafuwb
  3: xgzetva
  4: wfydsuz
  5: vexcrty
  6: udwbqsx
  7: tcvaprw
  8: sbuzoqv
  9: ratynpu
  10: qzsxmot
  11: pyrwlns
  12: oxqvkmr
  13: nwpujlq
  14: mvotikp
  15: lunshjo
  16: ktmrgin
  17: jslqfhm
  18: irkpegl
  19: hqjodfk
  20: gpincej
  21: fohmbdi
  \crim{22: englach}
  23: dmfkzbg
  24: clejyaf
  25: bkdixze
\end{Verbatim}
\end{multicols*}

\noindent
Much of the results doesn't make sense (as we expected), but one shift almost spells ``England''.

\noindent
Let's focus on the possibility of that being our keyword
--- in which case the last two characters we picked are likely wrong.

Using \Verb#``ENGLAND''# as the keyword, we get the following results:
\color{zaffre}
\begin{Verbatim}
  ITISTOBEQUESTIONEDWHETHERINTHEWHOLELENGTHANDBREADTHOF
  THEWORLDTHEREISAMOREADMIRABLESPOTFORAMANINLOVETOPASS
  ADAYORTWOTHANTHETYPICALENGLISHVILLAGEITCOMBINESTHE
  COMFORTSOFCIVILIZATIONWITHTHERESTFULNESSOFSOLITUDE
  INAMANNEREQUALLEDBYNOOTHERSPOTEXCEPTTHENEWYORKPUBLICLIBRARY
\end{Verbatim}
\color{black}

\noindent
When we space out and format the text properly, it reads:

\bigskip
\noindent
\color{teal}
It is to be questioned whether in the whole length and breadth
of the world there is a more admirable spot for a man in love
to pass a day or two than the typical English village.
It combines the comforts of civilization with the restfulness
of solitude in a manner equalled by no other spot except
the New York public library.
\color{black}
\end{Answer}
