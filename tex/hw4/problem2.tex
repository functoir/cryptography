\begin{problem}
Alice uses the Hill cipher, encrypting the plaintext
\begin{center}
\textsf{Consistency is the last refuge of the unimaginative}
\end{center}
to get the ciphertext
\begin{center}
\texttt{voqimugocogmttfkxvlvdynhawugtfrsksoizgaanlygk}
\end{center}
to send to Bob using blocks of size $m=3$ (and $n=26$).  Playing the
role of Eve, hack Alice's encryption key $A \in M_3(\Z/26\Z)$.  The
matrix key spells out a keyword: what is it?

After you find the key,
notice that Alice has not followed the protocol correctly.  Explain
why, then find two plaintexts that encrypt to the same ciphertext
using Alice's key.
\end{problem}

\begin{Answer}

Since we know the plaintext that was encrypted, we can use it (and the corresponding ciphertext)
to pull out equations from the encryption and find the key.

\noindent
Let $A = \big(\begin{smallmatrix}
  a_1 & a_2 & a_3 \\
  a_4 & a_5 & a_6 \\
  a_7 & a_8 & a_9
\end{smallmatrix}\big)$ be the encryption matrix.

\noindent
Then, we know that:

\begin{align*}
  \begin{bmatrix}
    a_1 & a_2 & a_3 \\
    a_4 & a_5 & a_6 \\
    a_7 & a_8 & a_9
  \end{bmatrix} \begin{bmatrix}
      c & t & a\\
      o & e & t\\
      n & n & i
    \end{bmatrix} &\equiv \begin{bmatrix}
      v & g & a\\
      o & o & n\\
      c & c & l
    \end{bmatrix} \pmod{26}\\
  \end{align*}
  \noindent
  We can now focus on the equations.
  \begin{align*}
    \begin{bmatrix}
      a_1 & a_2 & a_3 \\
      a_4 & a_5 & a_6 \\
      a_7 & a_8 & a_9
    \end{bmatrix} \begin{bmatrix}
        c & t & a\\
        o & e & t\\
        n & n & i
      \end{bmatrix} &\equiv \begin{bmatrix}
        v & g & a\\
        o & o & n\\
        c & c & l
    \end{bmatrix} \pmod{26}\\
    \begin{bmatrix}
      a_1 & a_2 & a_3 \\
      a_4 & a_5 & a_6 \\
      a_7 & a_8 & a_9
    \end{bmatrix} \begin{bmatrix}
         2 & 19 &  0\\
        14 &  4 & 19\\
        13 & 13 &  8
      \end{bmatrix}  &\equiv \begin{bmatrix}
      21 &  6 &  0\\
      14 & 14 & 13\\
      16 &  2 & 11 
    \end{bmatrix} \pmod{26}\\
    \begin{bmatrix}
      a_1 & a_2 & a_3 \\
      a_4 & a_5 & a_6 \\
      a_7 & a_8 & a_9
    \end{bmatrix} &\equiv \begin{bmatrix}
      21 &  6 &  0\\
      14 & 14 & 13\\
      16 &  2 & 11 
    \end{bmatrix} \begin{bmatrix}
       2 & 19 &  0\\
      14 &  4 & 19\\
      13 & 13 &  8
   \end{bmatrix}^{-1} \pmod{26}\\
   \begin{bmatrix}
    a_1 & a_2 & a_3 \\
    a_4 & a_5 & a_6 \\
    a_7 & a_8 & a_9
  \end{bmatrix} &\equiv \begin{bmatrix}
    21 &  6 &  0\\
    14 & 14 & 13\\
    16 &  2 & 11 
  \end{bmatrix} \begin{bmatrix}
   15 & 10 & 25\\
   19 & 14 & 22\\
    0 & 13 & 18
 \end{bmatrix} \pmod{26}\\
 \begin{bmatrix}
  a_1 & a_2 & a_3 \\
  a_4 & a_5 & a_6 \\
  a_7 & a_8 & a_9
\end{bmatrix} &\equiv \begin{bmatrix}
  13 &  8 &  7\\
   8 & 11 &  8\\
  18 & 19 & 18 
\end{bmatrix}\\
\begin{bmatrix}
  a_1 & a_2 & a_3 \\
  a_4 & a_5 & a_6 \\
  a_7 & a_8 & a_9
\end{bmatrix} &\equiv \begin{bmatrix}
   N & I & H\\
   I & L & I\\
   S & T & S 
\end{bmatrix}
\end{align*}
The matrix key spells out the word \verb#NIHILISTS#.
\newline

\noindent
Alice's encryption matrix is faulty because its determinant, $16$, is not coprime with $26$.
This means the matrix is not invertible since it is not injective---for
example, it maps both ``'' and ``'' to ``''. Bob needs to be able to
find the inverse in order to decrypt the message.

\end{Answer}
