\begin{problem}
The Hill cipher succumbs to a known plaintext attack if sufficiently
many plaintext-ciphertext pairs are known.
It is even easier to break the cipher if Eve can trick Alice into 
encrypting a chosen plaintext,
a \emph{chosen plaintext attack}.
Describe such an attack.
\end{problem}

\begin{Answer}

Although the Hill cipher generates somewhat ``contextualized'' ciphertext by having
each letter's encryption depend on a number of letters surrounding it,
the ciphertext it generates, when broken down into appropriate block sizes,
is always the same linear combination of the corresponding plaintext blocks.

\noindent
This is the weakness in the system---if enough plaintext and corresponding ciphertext
is known, then attackers (Eve, in this case) can compute the 

The Hill cipher produces the same linear combinations of the letter vectors it receives.
While this is helpful in avoiding single-letter encryption and ensuring that a letter's
encryption is, to an extent, also dependent on its surrounding letters,
it is also a weakness --- suppose a set of vectors and the corresponding set of ciphertexts
is known, then Eve can easily solve solve a linear system of equations to find the key.


\noindent
A chosen plaintext attack is a situation where an attacker (Eve, in this case)
tries to get the encryptor (Alice) to encrypt a specific word or words.
Using the original and encrypted forms of the words, Eve can generate the systems of equations
she needs to figure out the encryption matrix, henceforth breaking Alice's encryption.
\end{Answer}
