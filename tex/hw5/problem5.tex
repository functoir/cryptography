\begin{problem}
Put $f(X)=X^8+X^4+X^3+X+1 \in \F_2[X]$, and let 
\[ a=00001100=X^3+X^2 \in F=\F_2[X]/(f). \]  
\begin{enumerate}\renewcommand{\itemsep}{3mm}
\item Compute $a^5$.

\begin{Answer}
  
% \center{}
\begin{multicols}{2}

\crim{Let's begin by computing $a^2$:}
\newline
\color{zaffre}
\opmul[displayshiftintermediary=all]{1 1 0 0}{1 1 0 0}
$\equiv 1010000$
\color{black}

\noindent
\crim{We can then compute $a^4 = {(a^2)}^2$:}
\newline
\color{zaffre}
\opmul[displayshiftintermediary=all]{1 0 1 0 0 0 0}{1 0 1 0 0 0 0}
$\equiv 1 0 0 0 1 0 0 0 0 0 0 0 0$
\color{black}

\noindent
\crim{And, finally, $a^5 = a \cdot a^4$}
\newline
\color{zaffre}
\opmul[displayshiftintermediary=all]{1 0 0 0 1 0 0 0 0 0 0 0 0}{1 1 0 0}
\color{black}
\end{multicols}

\noindent
We then need to find the equivalent of $1 1 0 0 1 1 0 0 0 0 0 0 0 0 0 0$ in the Rijndael field $F$
by finding its modulus with $X^8+X^4+X^3+X+1$.

\bigskip
\noindent
\crim{Division by $x^8+x^4+x^3+x+1 \equiv 100011011$}.

\color{zaffre}
\begin{tabular}{c@{\,}c@{\,}c@{\,}c@{\,}c@{\,}c@{\,}c@{\,}c@{\,}c@{\,} | c@{\,}c@{\,}c@{\,}c@{\,}c@{\,}c@{\,}c@{\,}c@{\,}c@{\,}c@{\,}c@{\,}c@{\,}c@{\,}c@{\,}c@{\,}c@{\,}}
  1 & 0 & 0 & 0 & 1 & 1 & 0 & 1 & 1     & 1 & 1 & 0 & 0 & 1 & 1 & 0 & 0 & 0 & 0 & 0 & 0 & 0 & 0 & 0 & 0\\
\hline{}
    &   & 1 & 0 & 0 & 0 & 0 & 0 & 0     & 1 & 0 & 0 & 0 & 1 & 1 & 0 & 1 & 1 &.  &.  &.  &.  &.  &.  &.\\
    &   &   &   &   &   &   &   &       &   & 1 & 0 & 0 & 0 & 0 & 0 & 1 & 1 & 0 &.  &.  &.  &.  &.  &.\\
    &   &   & 1 & 0 & 0 & 0 & 0 & 0     &   & 1 & 0 & 0 & 0 & 1 & 1 & 0 & 1 & 1 &.  &.  &.  &.  &.  &.\\
    &   &   &   &   &   &   &   &       &   &   &   &   &   & 1 & 1 & 1 & 0 & 1 & 0 & 0 & 0 & 0 &.  &.\\
    &   &   &   &   &   & 1 & 0 & 0     &   &   &   &   &   & 1 & 0 & 0 & 0 & 1 & 1 & 0 & 1 & 1 &.  &.\\
    &   &   &   &   &   &   &   &       &   &   &   &   &   &   & 1 & 1 & 0 & 0 & 1 & 0 & 1 & 1 & 0 &.\\
    &   &   &   &   &   &   & 1 & 0     &   &   &   &   &   &   & 1 & 0 & 0 & 0 & 1 & 1 & 0 & 1 & 1 &.\\
    &   &   &   &   &   &   &   &       &   &   &   &   &   &   &   & 1 & 0 & 0 & 0 & 1 & 1 & 0 & 1 &0\\
    &   &   &   &   &   &   &   & 1     &   &   &   &   &   &   &   & 1 & 0 & 0 & 0 & 1 & 1 & 0 & 1 &1\\
\hline{}
    &   & 1 & 1 & 0 & 0 & 1 & 1 & 1     &   &   &   &   &   &   &   &   &   &   &   &   &   &   &   &1\\
\\
\\
\end{tabular}
\color{black}

\noindent
Thus, $a^5 \equiv 1 \in F$.
\end{Answer}

\newpage
\item Find the inverse $b^{-1} \in F$ of $b=X^2=00000100$.

\begin{Answer}

\noindent
Using the extended Euclidean Algorithm:

\noindent
$f(X)=X^8+X^4+X^3+X+1 = 100011011$

\begin{align*}
100011011  &\equiv 100 \cdot 1000110 + 11\\
100 &\equiv 11 \cdot 11 + 1\\
\bigskip\\
1 &\equiv 100 - 11 \cdot 11\\
1 &\equiv 100 - 11 \cdot (100011011-1000110 \cdot 100)\\
1 &\equiv 100 - 11(\crim{100011011}) + 11 \cdot 1000110 \cdot 100\\
1 &\equiv 100 + 11 \cdot 1000110 \cdot 100\\
1 &\equiv 100 + (100011011-1000110 \cdot 100) \cdot 1000110 \cdot 100\\
1 &\equiv 100 + \crim{100011011} \cdot 1000110 \cdot 100 - {(1000110 \cdot 100)}^{2}\\
1 &\equiv 100 - {(1000110 \cdot 100)}^{2}\\
1 &\equiv 100 - 1000110 \cdot 100 \cdot 1000110 \cdot 100\\
1 &\equiv 100 \cdot (1 - 100000001010000)\\
1 &\equiv 100 \cdot 100000001010001\\
1 &\equiv 100 \cdot 11001011 \text{ (see below for reduction)}\\
\therefore b^{-1} &\equiv 11001011\\
\
\end{align*}

\bigskip
\noindent
\crim{Division by $x^8+x^4+x^3+x+1 \equiv 100011011$}.

\bigskip
\color{zaffre}
\begin{tabular}{c@{\,}c@{\,}c@{\,}c@{\,}c@{\,}c@{\,}c@{\,}c@{\,}c@{\,} | c@{\,}c@{\,}c@{\,}c@{\,}c@{\,}c@{\,}c@{\,}c@{\,}c@{\,}c@{\,}c@{\,}c@{\,}c@{\,}c@{\,}c@{\,}c@{\,}}
  1 & 0 & 0 & 0 & 1 & 1 & 0 & 1 & 1     & 1 & 0 & 0 & 0 & 0 & 0 & 0 & 0 & 1 & 0 & 1 & 0 & 0 & 0 &1\\
\hline{}
    &   & 1 & 0 & 0 & 0 & 0 & 0 & 0     & 1 & 0 & 0 & 0 & 1 & 1 & 0 & 1 & 1 &.  &.  &.  &.  &.  &.\\
    &   &   &   &   &   &   &   &       &   &   &   &   & 1 & 1 & 0 & 1 & 0 & 0 & 1 & 0 & 0 &.  &.\\
    &   &   &   &   &   & 1 & 0 & 0     &   &   &   &   & 1 & 0 & 0 & 0 & 1 & 1 & 0 & 1 & 1 &.  &.\\
    &   &   &   &   &   &   &   &       &   &   &   &   &   & 1 & 0 & 1 & 1 & 1 & 1 & 1 & 1 & 0 &.\\
    &   &   &   &   &   &   & 1 & 0     &   &   &   &   &   & 1 & 0 & 0 & 0 & 1 & 1 & 0 & 1 & 1 &.\\
\hline{}
    &   & 1 & 0 & 0 & 0 & 1 & 1 & 0     &   &   &   &   &   &   &   & 1 & 1 & 0 & 0 & 1 & 0 & 1 &1\\
\\
\\
\end{tabular}
\color{black}

\end{Answer}
\newpage
\item Compute the product $b^{-1}a$ and verify that $b^{-1}a = X+1$ in $F$.
\begin{Answer}

\noindent
First, we need to find the product $b^{-1}a$:

\bigskip
\color{zaffre}
\opmul[displayshiftintermediary=all]{1 1 0 0 1 0 1 1}{1 1 0 0}
$\equiv 10101110100$
\color{black}

\bigskip

\noindent
We then find the equivalent of this product in $F$ by finding its modulus with

\noindent
$f(X)=X^8+X^4+X^3+X+1$

\bigskip
\color{zaffre}
\begin{tabular}{c@{\,}c@{\,}c@{\,}c@{\,}c@{\,}c@{\,}c@{\,}c@{\,}c@{\,} | c@{\,}c@{\,}c@{\,}c@{\,}c@{\,}c@{\,}c@{\,}c@{\,}c@{\,}c@{\,}c@{\,}c@{\,}c@{\,}c@{\,}c@{\,}c@{\,}}
  1 & 0 & 0 & 0 & 1 & 1 & 0 & 1 & 1     & 1 & 0 & 1 & 0 & 1 & 1 & 1 & 0 & 1 & 0 & 0\\
\hline{}
    &   &   &   &   &   & 1 & 0 & 0     & 1 & 0 & 0 & 0 & 1 & 1 & 0 & 1 & 1 &.  &.\\
    &   &   &   &   &   &   &   &       &   &   & 1 & 0 & 0 & 0 & 1 & 1 & 0 & 0 &0\\
    &   &   &   &   &   &   & 1 & 0     &   &   & 1 & 0 & 0 & 0 & 1 & 1 & 0 & 1 &1\\
\hline{}
    &   &   &   &   &   & 1 & 1 & 0     &   &   &   &   &   &   &   &   &   & 1 &1\\
\\
\\
\end{tabular}
\color{black}

\noindent
The modulus is $11$, equivalent to $X+1$.
\end{Answer}
\end{enumerate}
\end{problem}

