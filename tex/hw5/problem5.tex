\begin{problem}
Put $f(X)=X^8+X^4+X^3+X+1 \in \F_2[X]$, and let 
\[ a=00001100=X^3+X^2 \in F=\F_2[X]/(f). \]  
\begin{enumerate}\renewcommand{\itemsep}{3mm}
\item Compute $a^5$.

\begin{Answer}
  
% \center{}
\begin{multicols}{2}

\crim{Let's begin by computing $a^2$:}
\newline
\color{zaffre}
\opmul[displayshiftintermediary=all]{1 1 0 0}{1 1 0 0}
$\equiv 1010000$
\color{black}

\noindent
\crim{We can then compute $a^4 = {(a^2)}^2$:}
\newline
\color{zaffre}
\opmul[displayshiftintermediary=all]{1 0 1 0 0 0 0}{1 0 1 0 0 0 0}
$\equiv 1 0 0 0 1 0 0 0 0 0 0 0 0$
\color{black}

\noindent
\crim{And, finally, $a^5 = a \cdot a^4$}
\newline
\color{zaffre}
\opmul[displayshiftintermediary=all]{1 0 0 0 1 0 0 0 0 0 0 0 0}{1 1 0 0}
\color{black}
\end{multicols}

\noindent
We then need to find the equivalent of $1 1 0 0 1 1 0 0 0 0 0 0 0 0 0 0$ in the Rijndael field $F$
by finding its modulus with $X^8+X^4+X^3+X+1$.

\bigskip
\noindent
\crim{Division by $x^8+x^4+x^3+x+1 \equiv 100011011$}.

\color{zaffre}
\begin{tabular}{c@{\,}c@{\,}c@{\,}c@{\,}c@{\,}c@{\,}c@{\,}c@{\,}c@{\,} | c@{\,}c@{\,}c@{\,}c@{\,}c@{\,}c@{\,}c@{\,}c@{\,}c@{\,}c@{\,}c@{\,}c@{\,}c@{\,}c@{\,}c@{\,}c@{\,}}
  1 & 0 & 0 & 0 & 1 & 1 & 0 & 1 & 1     & 1 & 1 & 0 & 0 & 1 & 1 & 0 & 0 & 0 & 0 & 0 & 0 & 0 & 0 & 0 & 0\\
\hline{}
    &   & 1 & 0 & 0 & 0 & 0 & 0 & 0     & 1 & 0 & 0 & 0 & 1 & 1 & 0 & 1 & 1 &.  &.  &.  &.  &.  &.  &.\\
    &   &   &   &   &   &   &   &       &   & 1 & 0 & 0 & 0 & 0 & 0 & 1 & 1 & 0 &.  &.  &.  &.  &.  &.\\
    &   &   & 1 & 0 & 0 & 0 & 0 & 0     &   & 1 & 0 & 0 & 0 & 1 & 1 & 0 & 1 & 1 &.  &.  &.  &.  &.  &.\\
    &   &   &   &   &   &   &   &       &   &   &   &   &   & 1 & 1 & 1 & 0 & 1 & 0 & 0 & 0 & 0 &.  &.\\
    &   &   &   &   &   & 1 & 0 & 0     &   &   &   &   &   & 1 & 0 & 0 & 0 & 1 & 1 & 0 & 1 & 1 &.  &.\\
    &   &   &   &   &   &   &   &       &   &   &   &   &   &   & 1 & 1 & 0 & 0 & 1 & 0 & 1 & 1 & 0 &.\\
    &   &   &   &   &   &   & 1 & 0     &   &   &   &   &   &   & 1 & 0 & 0 & 0 & 1 & 1 & 0 & 1 & 1 &.\\
    &   &   &   &   &   &   &   &       &   &   &   &   &   &   &   & 1 & 0 & 0 & 0 & 1 & 1 & 0 & 1 &0\\
    &   &   &   &   &   &   &   & 1     &   &   &   &   &   &   &   & 1 & 0 & 0 & 0 & 1 & 1 & 0 & 1 &1\\
\hline{}
    &   & 1 & 1 & 0 & 0 & 1 & 1 & 1     &   &   &   &   &   &   &   &   &   &   &   &   &   &   &   &1\\
\\
\\
\end{tabular}
\color{black}

\noindent
Thus, $a^5 \equiv 1 \in F$.
\end{Answer}

\newpage
\item Find the inverse $b^{-1} \in F$ of $b=X^2=00000100$.

\begin{Answer}

For simplicity, I converted the binary-equivalent numbers to base 10 and
calculated the inverse using the extended euclidean algorithm.

\noindent
$b=X^2=00000100 \equiv 4$

\noindent
$f(X)=X^8+X^4+X^3+X+1 = 100011011 \equiv 283$

\begin{align*}
283 &\equiv 4 \cdot 70 + 3\\
70 &\equiv 3 \cdot 23 + 1\\
\\
1 &\equiv 70 - 3 \cdot 23\\
1 &\equiv 70 - 23 (283 - 4 \cdot 70)\\
1 &\equiv 70 - 283 \cdot 23 + 4 \cdot 23 \cdot 70\\
1 &\equiv -283 \cdot 23 + (4 \cdot 23 + 1) \cdot 70\\
1 &\equiv 93 \cdot 70 - 283 \cdot 23\\
1 &\equiv -213 \cdot 23
\end{align*}

We know, by definition:

\noindent
$b \cdot b^{-1} \equiv 1$

\noindent
$b^{-1} \equiv \frac{1}{b}$

\color{zaffre}
\begin{tabular}{c@{\,}c@{\,}c@{\,}c@{\,}c@{\,}c@{\,}c@{\,}c@{\,} | c@{\,}c@{\,}c@{\,}c@{\,}c@{\,}c@{\,}c@{\,}c@{\,}}
  0 & 0 & 0 & 0 & 0 & 1 & 0 & 0     & 0 & 0 & 0 & 0 & 0 & 0 & 0 & 1\\
\hline{}
    &   &   &   &   &   &   & 1     &   &   &   &   &   & 1 & 0 & 0\\
\hline{}
    &   &   &   &   &   &   & 1     &   &   &   &   &   & 1 & 0 & 1\\
\\
\\
\end{tabular}
\color{black}


  
\end{Answer}

\item Compute the product $b^{-1}a$ and verify that $b^{-1}a = X+1$ in $F$.
\begin{Answer}

\color{zaffre}
\opmul[displayshiftintermediary=all]{1 0 0 0 1 1 1}{1 1 0 0}
$\equiv 1100100100$


\bigskip

\color{zaffre}
\begin{tabular}{c@{\,}c@{\,}c@{\,}c@{\,}c@{\,}c@{\,}c@{\,}c@{\,}c@{\,} | c@{\,}c@{\,}c@{\,}c@{\,}c@{\,}c@{\,}c@{\,}c@{\,}c@{\,}c@{\,}c@{\,}c@{\,}c@{\,}c@{\,}c@{\,}c@{\,}}
  1 & 0 & 0 & 0 & 1 & 1 & 0 & 1 & 1     & 1 & 1 & 0 & 0 & 1 & 0 & 0 & 1 & 0 & 0\\
\hline{}
    &   &   &   &   &   &   & 1 & 0     & 1 & 0 & 0 & 0 & 1 & 1 & 0 & 1 & 1 &. \\
    &   &   &   &   &   &   &   &       &   & 1 & 0 & 0 & 0 & 1 & 0 & 0 & 1 & 0\\
    &   &   &   &   &   &   &   & 1     &   & 1 & 0 & 0 & 0 & 1 & 1 & 0 & 1 & 1\\
\hline{}
    &   &   &   &   &   &   &   &       &   &   &   &   &   &   & 1 & 1 & 0 & 1\\
\\
\\
\end{tabular}
\color{black}
\end{Answer}
\end{enumerate}
\end{problem}

