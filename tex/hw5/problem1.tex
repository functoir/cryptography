\begin{problem}
  Let $k \geq 2$ and $A=(\Z/2\Z)^k$.  Let $\vec{0}, \vec{1} \in A$ be
  the vectors of all zeros and all ones, respectively.  Define the map
  $g : A \to A$ by
\[
g(y) = 
\begin{cases}
\vec{0} & y \neq \vec{0} \\
\vec{1} & y = \vec{0} \\
\end{cases}
\]
Then define
\begin{align*} 
s,G: A \times A &\to A \times A \\
s(x,y) &= (y,x) \\
G(x,y) &= (x + g(y),y)
\end{align*}
\begin{enumerate}\renewcommand{\itemsep}{3mm}
\item Prove that $s^2$ and $G^2$ are the identity on $A \times A$.
\emph{[We actually proved this in lecture, so just make sure you
  understand it here.]}

\begin{Answer}

By definition:

\begin{align*}
  s(x, y) &= (y, x) \\
  s^2(x, y) = s(s(x, y)) &= s(y, x)\\
                         &= (x, y)
\end{align*}

\noindent
Likewise:

\begin{align*}
  G(x,y) &= (x+g(y), y)\\
  G^2(x,y) = G(x+g(y), y) &= (x+2g(y), y)\\
                          &\equiv (x+0, y) \pmod 2\\
                          &\equiv (x, y)
\end{align*}
\end{Answer}
\newpage
\item Prove that ${(sG)}^4=sgsgsgsg$ moves only $3$ elements of $A \times A$, i.e. 
\[ \#\{(x,y) \in A \times A:{(sG)}^4(x,y) \neq (x,y)\}=3. \]
\item Prove that ${(sG)}^{12}$ is the identity.
\end{enumerate}

\begin{Answer}
We know that:
\begin{align*}
  G(x,y) &= (x + g(y), y)\\
  s(x,y) &= (y,x)\\
  \therefore sG(x,y) &= (y, x+g(y)) 
\end{align*}

\noindent
For the case of simplicity,
let's substitute $0$ and $1$ for $x, y$ where necessary
and retain $x$, $y$ where the values of $x$ and $y$ are not $\vec{0}$ or $\vec{1}$.

\noindent
Then, each vriable can assume any of three general values:
$\vec{0}$, $\vec{1}$, or an intermediate value (such as the vector $\left<1, 0, 0, 1\right>$),
which will be represented as just $x$ or $y$.

\centering
\begin{tabular}{||c | c | c | c | c | c | c||} 
\hline
  $\id{}$ & $sG$ & ${(sG)}^{2}$ & ${(sG)}^{3}$ &  ${(sG)}^{4}$ & ${(sG)}^{8}$ & ${(sG)}^{12}$\\ [0.5ex] 
\hline\hline
  (0,0) & (0,1)     & (1,0)     & (0,0)     & \crim{(0,1)} & (1,0) & (0,0)\\ 
  (0,y) & (y,0)     & (0,\^{y}) & (\^{y},0) & (0,y)        & (0,y) & (0,y)\\
  (0,1) & (1,0)     & (0,0)     & (0,1)     & \crim{(1,0)} & (0,0) & (0,1)\\
  (x,0) & (0,\^{x}) & (\^{x},0) & (0,x)     & (x,0)        & (x,0) & (x,0)\\ 
  (x,y) & (y,x)     & (x,y)     & (y,x)     & (x,y)        & (x,y) & (x,y)\\
  (x,1) & (1,x)     & (x,1)     & (1,x)     & (x,1)        & (x,1) & (x,1)\\
  (1,0) & (0,0)     & (0,1)     & (1,0)     & \crim{(0,0)} & (0,1) & (1,0)\\ 
  (1,y) & (y,1)     & (1,y)     & (y,1)     & (1,y)        & (1,y) & (1,y)\\
  (1,1) & (1,1)     & (1,1)     & (1,1)     & (1,1)        & (1,1) & (1,1)\\[1ex]
\hline
\end{tabular}

\flushleft{}
\bigskip
From the table, we can infer that:

\begin{enumerate}
  \item $sG(x)$ is the identity for $x \in \left\{ (1,1) \right\}$.
  \item $sG(x)$ repeats values with a period of $2$ for $x \in \left\{ (x,1), (1,y), (x,y)\right\}$.
  \item $sG(x)$ repeats values with a period of $3$ for $x \in \left\{ (0,0), (0,1) (1,0) \right\}$.
  \item $sG(x)$ repeats values with a period of $4$ for $x \in \left\{ (x,0), (0,y) \right\}$.
\end{enumerate}

\bigskip
\color{zaffre}
\noindent
Thus, $sG^{4}$ will be the identity for all values where
  the sequence has a period that is a divisor of $4$.
  These are all values \textbf{except} $\left\{ (0,0), (0,1), (1,0) \right\}$,
  which have a period of $3$ and $3 \nmid 4$.
  Indeed, as we can see from the table, ${(sG)}^{4}$ is fixed for all except these.

\bigskip
On the other hand, $(1 \mid 12) \wedge (2 \mid 12) \wedge (3 \mid 12) \wedge (4 \mid 12)$.
  Therefore, ${(sG)}^{12}(x,y)$ will be the identity for all $(x,y)$. 

\color{black}
\end{Answer}

\end{problem}
